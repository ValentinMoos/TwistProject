\ifdefined\mainprogram{}
\else
\documentclass[10pt]{article}

%packages +  bib
\usepackage[left=2.5cm,right=2.5cm,bottom=2cm,top=3.0cm]{geometry}
\usepackage{amsmath}      %std package for many operators
\usepackage{amssymb}      %symbolds i guess
\usepackage{bbold}	  %identity matrix
\usepackage{ dsfont }	  % i have no idea
\usepackage{color}              % for comments
\usepackage{xcolor}	   %for additional color, can be deleted in the end i think!!!s
\usepackage{physics}          % bra ket notation
\usepackage{slashed}          %for slash notation
\usepackage{fancyhdr}       %allows chance of layout
\usepackage[subfigure]{tocloft}    %TOC layout
\usepackage{indentfirst}      %for parindent after new section
\usepackage{graphicx,subfigure}	  %for uni logo
\usepackage{setspace}      %for spacing between the lines
\usepackage{simplewick}   %for wick contractions
\usepackage{tikz-feynman} %SM diagrams
\usepackage[title,titletoc]{appendix} %appendix written in table of contents
\usepackage[sorting=none]{biblatex}  %numbers of citation are how they appear in tex
\bibliography{literature.bib}

%Layout including how to enumberate equations
\setlength\parindent{0.8cm}
\setlength{\cftsecnumwidth}{35pt}
\setlength{\cftsubsecnumwidth}{35pt}
\pagestyle{fancy} %benutzerdef
\fancyhf{}
\fancyhead[R]{\thepage}
\renewcommand{\headrulewidth}{0pt}

\renewcommand{\thesection}{\Roman{section}} % i removed the ``.'' in the end ( in subsection it is still there) for references. It should work this way.
\renewcommand{\thesubsection}{\Alph{subsection}}
\renewcommand{\thesubsubsection}{\Alph{subsection}.\roman{subsubsection}}



\numberwithin{equation}{section}
\renewcommand{\theequation}{\arabic{section}.\arabic{equation}}

\renewcommand{\baselinestretch}{1.25}

%Additional Layout matters
\newcommand{\remark}[1]{\newline\newline \emph{#1} \newline\newline}

%dates


%Other / Names
\newcommand{\Mat}{\text{\emph{Mathematica}}}


%References
\newcommand{\howtowriteequation}{eq.$~$}
\newcommand{\cref}[1]{eq.$~$(#1)} % cite equation from other paper ref
\newcommand{\eref}[1]{\howtowriteequation (\ref{#1})}
\newcommand{\doubleref}[2]{\howtowriteequation (\ref{#1}, \ref{#2})}
\newcommand{\tripleref}[3]{\howtowriteequation (\ref{#1}, \ref{#2}, \ref{#3})}
\newcommand{\ereffromto}[2]{\howtowriteequation (\ref{#1})-(\ref{#2})}

\newcommand{\sref}[1]{section$~$\ref{#1}}
\newcommand{\Sref}[1]{Section$~$\ref{#1}}
\newcommand{\tref}[1]{table$~$\ref{#1}}
\newcommand{\fref}[1]{figure$~$\ref{#1}}
\newcommand{\aref}[1]{appendix$~$\ref{#1}}

\newcommand{\pref}[1]{page$~$\pageref{#1}}

\newcommand{\mc}[1]{\cite{#1}}

%Notes for correcting
\newcommand{\COM}[1]{\text{\textcolor{red}{#1}}}
\newcommand{\LBL}[0]{\COM{Hier}}
\newcommand{\spacee}{~~~~~~~}
%newline in formula with term exeeding length of a line
\newcommand{\nl}{\\&~~~}

%Functions and symbols
\newcommand{\gf}{G}
\newcommand{\hamiltonian}{H}
\newcommand{\flnso}{Op.}
\newcommand{\propagator}{\Delta}  % propagator symbol

%Fields (quark, gauge etc)
\newcommand{\Aqu}{B} %quantum gauge field
%operators
\newcommand{\operatorO}{O}
\newcommand{\twoq}[1]{\operatorO_q^{#1}}
\newcommand{\twoqbar}[1]{\overline \operatorO_{q}^{#1}}
\newcommand{\twog}[1]{\operatorO_{g}^{#1}}
\newcommand{\twogbar}[1]{\overline \operatorO_{g}^{#1}}

%Differential Operators
\newcommand{\dx}{\text{d}}
\newcommand{\abdif}[1]{\frac{\dx}{\dx #1  } }
\newcommand{\abdifof}[2]{\frac{\dx #1}{\dx #2  } }
\newcommand{\padif}[1]{\frac{\partial}{\partial #1  } }
\newcommand{\doublepadif}[2]{\frac{\partial^2}{\partial #1 \partial #2}}
\newcommand{\padifntimes}[2]{\frac{\partial^{#2}}{\partial #1^{#2}}}
\newcommand{\ft}{F}%field tensor
\newcommand{\cov} {D}
\newcommand{\covleft}{\overleftarrow{\cov}}
\newcommand{\covright}{\overrightarrow{\cov}}
\newcommand{\dxabc}{\left[\dx \alpha \dx \beta \dx \gamma\right]}
\newcommand{\dxabcprime}{\left[\dx \alpha' \dx \beta' \dx \gamma'\right]}
\newcommand{\dxfm}[2]{\frac{\dx^{#2}#1}{\left(2\pi\right)^{#2}}}
\newcommand{\dxfs}[2]{\dx^{#2}#1}
\newcommand{\ddx}{\dx ^d}


%Michelaneous including spinors
\newcommand{\metric}{\eta}
\newcommand{\idm}{{1}}
\newcommand{\abelian}[2]{\left[ #1, #2 \right]}
\newcommand{\expo}{e}
\newcommand{\skp}[2]{(#1|#2)}
\newcommand{\skpt}[2]{(#1\cdot#2)}
\newcommand{\spp}[1]{\langle #1 \rangle}
\newcommand{\aspp}[1]{\left[ #1 \right]}
\renewcommand{\trace}[0]{\text{Tr}}

%spinors
\newcommand{\psibar}{\overline\psi}
\newcommand{\psibarc}{\psibar_c}
\newcommand{\psibarq}{\psibar_q}
\newcommand{\psic}{\psi_c}
\newcommand{\psiq}{\psi_q}
\newcommand{\psip}{\psi_{+}}
\newcommand{\psim}{\psi_{-}}
\newcommand{\psit}{\overline\psi}
\newcommand{\chit}{\overline\chi}
\newcommand{\psipt}{\psit_{+}}
\newcommand{\psimt}{\psit_{-}}
\newcommand{\chip}{\chi_{+}}
\newcommand{\chim}{\chi_{-}}
\newcommand{\chipt}{\chit_{+}}
\newcommand{\chimt}{\chit_{-}}
\newcommand{\ispace}{~\hspace{-3pt}}
\newcommand{\ospex}{O}
\newcommand{\spideriv}[2]{\frac{#1\partial}{\partial #2}}
\newcommand{\spiderivb}[3]{\left(\spideriv{#1}{#2}\right)^{#3}}
\newcommand{\fbar}{\overline f}
\newcommand{\SX}{J}

%%Spinor Combinations
\newcommand{\lol}{\lambda\overline\lambda}
\newcommand{\mom}{\mu\overline\mu}
\newcommand{\lom}{\lambda\overline\mu}
\newcommand{\mol}{\mu\overline\lambda}
\newcommand{\olom}{\overline\lambda\overline\mu}
\newcommand{\omol}{\overline\mu\overline\lambda}
\newcommand{\om}{\overline\mu}
\newcommand{\ol}{\overline\lambda}
%skp that appear quite often such that the order is consistent
\newcommand{\skpyP}{\skp{y}{P}}
\newcommand{\skpPy}{\skpyP}

\newcommand{\skpyS}{\skp{y}{S}}
\newcommand{\skpSy}{\skpyS}
% q Bar, bold B , pplus, nbar
\newcommand{\qbar}{\overline q}
\newcommand{\bb}{\boldsymbol{b}}
\newcommand{\pp}{p_{+}}
\newcommand{\pphat}{\hat p_{+}}
\newcommand{\nbar}{\tilde n}
%Imaginary Unity, coupling: 
\newcommand{\imag}{\text{i}}
\newcommand{\as}{\alpha_s}
%Operators
%Wilson line, pretzelosity , Matrix element PS
\newcommand{\brez}{h_{1T}^{\bot}}
\newcommand{\wil}[2]{\left[#1,#2 \right]}
\newcommand{\myState}{P,S}
\newcommand{\fme}[1]{\bra{\myState}#1\ket{\myState}}
\newcommand{\twist}{T}
%O Gamma operators
\newcommand{\oga}{O ^{\Gamma  }}
\newcommand{\ogmu}{O^{\gamma^\mu}}
\newcommand{\ogmugfive}{O^{\gamma^\mu \gamma^5}}
\newcommand{\osigma}{O^{\imag \sigma ^{\mu \nu }\gamma_5}}
\newcommand{\ogplus}{O^{\gamma^{+}}}
\newcommand{\ogplusgfive}{O^{\gamma^{+} \gamma^5}}
\newcommand{\osigmaplus}{O^{\imag \sigma ^{\alpha + }\gamma_5}}
\newcommand{\osigmavarplus}[1]{O^{\imag \sigma ^{#1 + }\gamma_5}}
%3point operators
\newcommand{\ttp}{\mathcal{T}^\Gamma}
\newcommand{\ttpg}{\mathcal{T}^{\gamma_+}}
\newcommand{\ttpgg}{\mathcal{T}^{\gamma_+\gamma_5}}
\newcommand{\ttpsg}{\mathcal{T}^{\imag \sigma^{\alpha + }\gamma^5}}
\newcommand{\ttpsgadaptive}[3]{\mathcal{T}^{\imag \sigma^{#1 #2}\gamma^5}_{#3}}
%matrix elements of 3 point operators
\newcommand{\Dt}{\Delta T}
\newcommand{\Dtx}{\Dt (x_1,x_2,x_3)}
\newcommand{\Dtt}{\Delta \tilde T}
\newcommand{\dttg}{\delta \tilde T_g}
\newcommand{\dtte}{\delta \tilde T_\expsilon}
\newcommand{\dTe}{\delta T _\epsilon}
\newcommand{\dTex}{\delta T _\epsilon(x_1,x_2,x_3)}
\newcommand{\dTg}{\delta T _g}
\newcommand{\dTgx}{\delta T _g(x_1,x_2,x_3)}
\newcommand{\xonetwothree}{x_{1,2,3}}
%U gamma operators or full TMD s
\newcommand{\uga}{\mathcal{U}^{\Gamma}}
\newcommand{\udis}{\mathcal{U}_{DIS}^{\Gamma}}
\newcommand{\udy}{\mathcal{U}_{DY}^{\Gamma}}
\newcommand{\udygplus}{\mathcal{U}_{DY}^{\gamma^+}}
\newcommand{\udygplusgfive}{\mathcal{U}_{DY}^{\gamma^+\gamma_5}}
\newcommand{\udysg}{\mathcal{U}_{DY}^{\imag \sigma^{\alpha +}_{T} \gamma_5}}

\newcommand{\nameofO}{O}	
\newcommand{\ogt}{\nameofO_{TMD}^{\Gamma}}
\newcommand{\ogtfields}[2]{\nameofO_{#1}^{#2}}
\newcommand{\ogpt}{\nameofO_{TMD}^{\gamma_+}}

%phi distributions
\newcommand{\phiqh}{\Phi_{q\leftarrow h}}
\newcommand{\phiqhij}{\Phi_{q\leftarrow h,ij}}
\newcommand{\phiG}{\Phi_{q\leftarrow h}^{[\Gamma]}}
\newcommand{\phig}{\Phi_{q\leftarrow h}^{[\gamma^+]}}
\newcommand{\phigg}{\Phi_{q\leftarrow h}^{[\gamma^+ \gamma_5]}}
\newcommand{\phisg}[1]{\Phi_{q\leftarrow h}^{[\sigma^{#1 +}\gamma_5]}}

%Parametrization fucntions
\newcommand{\paraA}{A}
\newcommand{\paraB}{B}


%Graphics

\usepackage[]{hyperref}

\begin{document}

\fi

\subsection{gluons}

A better working method is maybe to parallel write serious exressions and then drop from them from time to time parts into ``prefactor'' expression

Let me do parallel computing. 

I take color red for right field, color blue for left field. Common factors i put out in front will be black, no color.
Thus here we have the following diagrams.
{\color{red}
\begin{align}
	&\contraction[3ex]{\bigg(-\imag g \int \ddx xA_\nu^D(x) \partial_{x^\alpha}}{B}{_\beta^{E} (x) B_\gamma^F(x) \bigg) v^{\nu \alpha\beta\gamma}_{DEF}\imag g \int_{-z}^{-\infty} \dx \sigma B_+^{B'}(\sigma n) T^{B'}_{CA'}\partial_\mu}{B}
	\contraction[2ex]{\bigg(-\imag g \int \ddx xA_\nu^D(x) \partial_{x^\alpha} B_\beta^{E} (x) }{B}{_\gamma^F(x) \bigg) v^{\nu \alpha\beta\gamma}_{DEF}\imag g \int_{-z}^{-\infty} \dx \sigma }{B}
	\bigg(-\imag g \int \ddx xA_\nu^D(x) \partial_{x^\alpha} B_\beta^{E} (x) B_\gamma^F(x) \bigg) v^{\nu \alpha\beta\gamma}_{DEF}\imag g \int_{-z}^{-\infty} \dx \sigma B_+^{B'}(\sigma n) T^{B'}_{CA'}\partial_\mu B^{A'}_\rho (-zn) 
	\nl
	+
	\contraction[3ex]{\bigg(-\imag g \int \ddx xA_\nu^D(x) \partial_{x^\alpha}}{B}{_\beta^{E} (x) B_\gamma^F(x) \bigg) v^{\nu \alpha\beta\gamma}_{DEF}\imag g \int_{-z}^{-\infty} \dx \sigma}{B}
	\contraction[2ex]{\bigg(-\imag g \int \ddx xA_\nu^D(x) \partial_{x^\alpha} B_\beta^{E} (x)}{B}{_\gamma^F(x) \bigg) v^{\nu \alpha\beta\gamma}_{DEF}\imag g \int_{-z}^{-\infty} \dx \sigma B_+^{B'}(\sigma n) T^{B'}_{CA'}\partial_\mu}{B}
	\bigg(-\imag g \int \ddx xA_\nu^D(x) \partial_{x^\alpha} B_\beta^{E} (x) B_\gamma^F(x) \bigg) v^{\nu \alpha\beta\gamma}_{DEF}\imag g \int_{-z}^{-\infty} \dx \sigma B_+^{B'}(\sigma n) T^{B'}_{CA'}\partial_\mu B^{A'}_\rho (-zn)
\end{align}
}
{\color{blue}
\begin{align}
&\contraction[2ex]{\partial_\mu}{B}{^{A'}_\rho(zn) \imag g \int _{-\infty}^{z} \dx \sigma B_+^{B'}(\sigma n) T^{B'}_{A'C}\bigg(-\imag g \int \ddx xA_\nu^D(x) \partial_{x^\alpha}}{B}
\contraction[3ex]{\partial_\mu B^{A'}_\rho (zn) \imag g \int _{-\infty}^{z} \dx \sigma}{B}{_+^{B'}(\sigma n) T^{B'}_{A'C}\bigg(-\imag g \int \ddx xA_\nu^D(x) \partial_{x^\alpha} B_\beta^{E} (x) }{B}
\partial_\mu B^{A'}_\rho (zn) \imag g \int _{-\infty}^{z} \dx \sigma B_+^{B'}(\sigma n) T^{B'}_{A'C}\bigg(-\imag g \int \ddx xA_\nu^D(x) \partial_{x^\alpha} B_\beta^{E} (x) B_\gamma^F(x) \bigg) v^{\nu \alpha\beta\gamma}_{DEF}
\nl
+\contraction[3ex]{\partial_\mu}{B}{^{A'}_\rho (zn) \imag g \int _{-\infty}^{z} \dx \sigma B_+^{B'}(\sigma n) T^{B'}_{A'C}\bigg(-\imag g \int \ddx xA_\nu^D(x) \partial_{x^\alpha} B_\beta^{E} (x) }{B}
\contraction[2ex]{\partial_\mu B^{A'}_\rho (zn) \imag g \int _{-\infty}^{z} \dx \sigma}{B}{_+^{B'}(\sigma n) T^{B'}_{A'C}\bigg(-\imag g \int \ddx xA_\nu^D(x) \partial_{x^\alpha} }{B}
\partial_\mu B^{A'}_\rho (zn) \imag g \int _{-\infty}^{z} \dx \sigma B_+^{B'}(\sigma n) T^{B'}_{A'C}\bigg(-\imag g \int \ddx xA_\nu^D(x) \partial_{x^\alpha} B_\beta^{E} (x) B_\gamma^F(x) \bigg) v^{\nu \alpha\beta\gamma}_{DEF}
\end{align}
}
For the colors:
{\color{blue}
\begin{align}
f^{DEF} T^{B'}_{A'C}(\delta^{A'E} \delta^{B'F} + \delta^{A'F} \delta^{B'E})
&=
-\imag(f^{DEF}f^{B'A'C}\delta^{A'E} \delta^{B'F} + f^{DEF}f^{B'A'C}\delta^{A'F} \delta^{B'E})
\\
&=
-\imag(-f^{DEF}f^{CEF} + f^{DEF}f^{CEF})
\\
&=
\imag C_A\delta^{CD}(+1 -1)
\end{align}
}
$+1-1$ means, the first line has $+$, the second line has $-1$.
 {\color{red}
\begin{align}
f^{DEF} T^{B'}_{CA'}(\delta^{A'E} \delta^{B'F} + \delta^{A'F} \delta^{B'E})
&=
-\imag(f^{DEF}f^{B'CA'}\delta^{A'E} \delta^{B'F} + f^{DEF}f^{B'CA'}\delta^{A'F} \delta^{B'E})
\\
&=
-\imag(f^{DEF}f^{FCE} + f^{DEF}f^{ECF})
\\
&=
-\imag f^{DEF}f^{FCE}(+1-1)
\\
&=
-\imag C_A\delta^{CD}(+1 -1)
\\
&=
-{\color{blue} \imag C_A\delta^{CD}(+1 -1)}
\end{align}
}
so from color the two diagrams differ by a total sign
First common factors:
\begin{align}
	+\imag^3 C_A g^2 \int \ddx  v^{\nu \alpha\beta\gamma}\delta^{CD}
\end{align}


\begin{align}
	&{\color{blue}\int_{-\infty}^{z}\partial_{zn^\mu} \partial_{x^\alpha} \propagator(zn-x)_{\rho \beta} \propagator(x-\sigma n)_{\gamma +} - \partial_{zn^\mu}\propagator(zn-x)_{\rho \gamma}\partial_{x^\alpha}  \propagator(x-\sigma n)_{\beta +}}
	\\
	&-{\color{red}\int_{-z}^{-\infty}\partial_{zn^\mu} \partial_{x^\alpha} \propagator(zn+x)_{\rho \beta} \propagator(x-\sigma n)_{\gamma +} - \partial_{zn^\mu}\propagator(zn+x)_{\rho \gamma}\partial_{x^\alpha}  \propagator(x-\sigma n)_{\beta +}}
\end{align}
Then we have the to compute as usual, derivatives on propagators. Since i know that the metric computation will kill the additional term from double derivative on one propagator (additional term $\tilde \metric_{\mu \alpha}$ I IGNORE IT HERE.)
\begin{align}
	&+\imag^3 C_A g^2 \int \ddx \delta^{CD}\Gamma^2\left(\frac{d}{2}-1\right)\frac{1}{4^2\pi^d}
\end{align}
{\color{blue}
\begin{align}
	&\int_{-\infty}^{z}\dx \sigma v^{\nu \alpha\beta\gamma}
	\nl
	\frac{4(\frac{d}{2}-1)\frac{d}{2}(zn-x)_\mu(x-zn)_\alpha}{\left(-(x-zn)^2 + \imag \epsilon\right)^{\frac{d}{2}+1}} 
	\frac{1}{\left(-(x-\sigma n)^2 + \imag \epsilon\right)^{\frac{d}{2}-1}}\metric_{\rho \beta} \metric_{\gamma +} 
	\nl
	- \frac{-2(\frac{d}{2}-1)(zn-x)_\mu}{\left( -(x-z n)^2 + \imag \epsilon \right)^{\frac{d}{2}}}\frac{-2(\frac{d}{2}-1)(x-\sigma n)_\alpha}{\left(-(x-\sigma n)^2 + \imag \epsilon\right)^{\frac{d}{2}}}
	\metric_{\rho \gamma}\metric_{\beta +}
\end{align}
}

{\color{red}
\begin{align}
	&-\int_{-z}^{-\infty}\dx \sigma v^{\nu \alpha\beta\gamma}
	\nl
	\frac{4(\frac{d}{2}-1)\frac{d}{2}(-zn-x)_\mu(x+zn)_\alpha}{\left(-(x+zn)^2 + \imag \epsilon\right)^{\frac{d}{2}+1}} 
	\frac{1}{\left(-(x-\sigma n)^2 + \imag \epsilon\right)^{\frac{d}{2}-1}}\metric_{\rho \beta} \metric_{\gamma +} 
	\nl
	- \frac{-2(\frac{d}{2}-1)(-zn-x)_\mu}{\left( -(x+z n)^2 + \imag \epsilon \right)^{\frac{d}{2}}}\frac{-2(\frac{d}{2}-1)(x-\sigma n)_\alpha}{\left(-(x-\sigma n)^2 + \imag \epsilon\right)^{\frac{d}{2}}}
	\metric_{\rho \gamma}\metric_{\beta +}
\end{align}
}
shifts:
{\color{blue}
\begin{align}
	x\rightarrow &x+ uzn + \overline u  \sigma n\\
	\sigma \rightarrow& \sigma +z
	\\
	HENCE\\
	x\rightarrow & x+ zn + \overline u \sigma n
	\\
	x-zn \rightarrow &  x + \overline u \sigma n
	\\
	x-\sigma n \rightarrow & x  - u \sigma n
\end{align}
}
{\color{red}
\begin{align}
	x\rightarrow &x- uzn + \overline u  \sigma n\\
	\sigma \rightarrow& \sigma -z
	\\
	HENCE\\
	x\rightarrow & x- zn + \overline u \sigma n
	\\
	x+zn \rightarrow &  x + \overline u \sigma n
	\\
	x-\sigma n \rightarrow & x - u \sigma n
\end{align}
}
Lorentz index computation is the same for both, red and blue. it is
\begin{align}
v^{\nu\alpha\beta\gamma} 
&=
2\metric^{\nu \beta}\metric^{\alpha\gamma}-\metric^{\nu \alpha}\metric^{\beta \gamma} -2\metric^{\nu \gamma} \metric^{\alpha\beta}
\\
v^{\nu\alpha\beta\gamma}\metric_{\mu\alpha}\metric_{\rho\beta}\metric_{+\gamma}
&=
0\text{(that is why there is no additional term from the double derivative on prop.)}
\\
v^{\nu\alpha\beta\gamma}\metric_{\rho\beta}\metric_{+\gamma}
&=
2(\metric^{\nu \rho} \metric^{\alpha+} - \metric^{\nu +}\metric^{\alpha\rho})
\\
v^{\nu\alpha\beta\gamma}\metric_{\rho\gamma}\metric_{+\beta}
&=
-2(\metric^{\nu \rho} \metric^{\alpha+} - \metric^{\nu +}\metric^{\alpha\rho})
\end{align}

and results in relative minus sign between the respective lines plus a common factor:
\begin{align}
	&+\imag^3 C_A g^2 \int \ddx \delta^{CD}\Gamma^2\left(\frac{d}{2}-1\right)\frac{1}{4^2\pi^d}2(\metric^{\nu \rho} \metric^{\alpha+} - \metric^{\nu +}\metric^{\alpha\rho})
\end{align}
The minus sign from this computation annihilates the relative minus from color (the minus between the diagrams! this has nothing to do with the minus between red and blue!)
{\color{blue}
\begin{align}
	&\int_{-\infty}^{0}\int_0^1 \dx \frac{\Gamma(d)}{\Gamma^2\left(\frac{d}{2}-1\right)} u^{\frac{d}{2}-1}\overline u ^{\frac{d}{2}-2}(-4x_\mu)\frac{u(x+\overline u \sigma n)_\alpha + \overline u(x- u\sigma n)_\alpha}{(-x^2+\imag \epsilon)^{d}}
	\\
	&=
	-4\int_{-\infty}^{0}\int_0^1 \dx \frac{\Gamma(d)}{\Gamma^2\left(\frac{d}{2}-1\right)} u^{\frac{d}{2}-1}\overline u ^{\frac{d}{2}-2}\frac{x_\mu x_\alpha}{(-x^2+\imag \epsilon)^{d}}
\end{align}
}

{\color{red}
\begin{align}
	&-\int_{0}^{-\infty}\int_0^1 \dx \frac{\Gamma(d)}{\Gamma^2\left(\frac{d}{2}-1\right)} u^{\frac{d}{2}-1}\overline u ^{\frac{d}{2}-2}(-4x_\mu)\frac{u(x+\overline u \sigma n)_\alpha + \overline u(x- u\sigma n)_\alpha}{(-x^2+\imag \epsilon)^{d}}
	\\
	&=
	+4\int_{0}^{-\infty}\int_0^1 \dx  \frac{\Gamma(d)}{\Gamma^2\left(\frac{d}{2}-1\right)} u^{\frac{d}{2}-1}\overline u ^{\frac{d}{2}-2}\frac{x_\mu x_\alpha}{(-x^2+\imag \epsilon)^{d}}
	\\
	&=
	-4\int_{-\infty}^{0}\int_0^1 \dx \frac{\Gamma(d)}{\Gamma^2\left(\frac{d}{2}-1\right)} u^{\frac{d}{2}-1}\overline u ^{\frac{d}{2}-2}\frac{x_\mu x_\alpha}{(-x^2+\imag \epsilon)^{d}}
\end{align}
}
at this point it looks like both computations are identical, but there is still the free field $A_\nu(x)$, which after shift is
\begin{align}
	{\color{blue}A(x+zn+\overline u \sigma n)} \spacee {\color{red}A(x-zn+\overline u \sigma n)}
\end{align}
This has to be kept in mind. 
Further the integration can be done in for both terms identically. I write $A(x+yn+\overline u \sigma n)$, and $y=\pm z$, blue or red case. It follows the following computation:
\begin{align}
	&+\imag^3 C_A g^2 \delta^{CD}\Gamma^2\left(\frac{d}{2}-1\right)\frac{1}{4^2\pi^d}2(\metric^{\nu \rho} \metric^{\alpha+} - \metric^{\nu +}\metric^{\alpha\rho})(-4)\int_{-\infty}^{0}\int_0^1 \dx \frac{\Gamma(d)}{\Gamma^2\left(\frac{d}{2}-1\right)} u^{\frac{d}{2}-1}\overline u ^{\frac{d}{2}-2}
\end{align}
\begin{align}
	 \int \ddx \frac{x_\mu x_\alpha x_\tau x_\sigma}{(-x^2+\imag \epsilon)^{d}}\partial^\tau \partial^\sigma A(yn+\overline u \sigma n) 
	 &= 
	\frac{\imag \pi^{\frac{d}{2}}}{4(2-\frac{d}{2})\Gamma(d)}\left(  \metric_{\mu \alpha} \metric_{\tau \sigma}+ \metric_{\mu \tau} \metric_{\alpha\sigma} + \metric_{\mu \sigma} \metric_{\alpha \tau} \right) A_\nu(yn+\overline u \sigma n)
\end{align}
Together metric  and fields give
\begin{align}
	&\left(  \metric_{\mu \alpha} \metric_{\tau \sigma}+ \metric_{\mu \tau} \metric_{\alpha\sigma} + \metric_{\mu \sigma} \metric_{\alpha \tau} \right) A_\nu(yn+\overline u \sigma n){\color{purple}- (\mu \leftrightarrow \rho)  [\rho = +]}
	\\
	&=2 \partial_+ \left( \partial_\mu A_+ - \partial_+ A_\mu \right)(yn+\overline u \sigma n)
\end{align}

Together with prefactors and so on

\begin{align}
	\frac{ (-4)2^2\imag^4 C_A g^2 }{4^3\pi^{\frac{d}{2}}\epsilon} \int_{-\infty}^{0}\dx \sigma\int_0^1 \dx u u^{\frac{d}{2}-1}\overline u ^{\frac{d}{2}-2} \partial_+\left( \partial_\mu A^C_+ - \partial_+ A^C_\mu \right)(yn+\overline u \sigma n)
\end{align}

now integration: 
\begin{align}
	&
	\int_{0}^{-\infty} \dx \sigma \int_0^1 \dx u u \partial_+ f(yn+\overline u \sigma n)
	\\
	&=
	\int \frac{\ddx p}{(2\pi)^4}\int_{0}^{-\infty} \dx \sigma \int_0^1 \dx u u (-\imag\pp) \expo^{\imag \skp{y+\overline u \sigma n}{p}+\delta \sigma} f(p)
	\\
	&=
	-\int \frac{\ddx p}{(2\pi)^4} \int_0^1 \dx u u \frac{-\imag\pp}{\imag \overline u \pp + \delta} \expo^{\imag \skp{y}{p}} f(p)
	\\
	&=
	\int \frac{\ddx p}{(2\pi)^4}\int_0^1 \dx \overline u \frac{1-\overline u}{ \overline u + \frac{\delta}{\imag\pp}} \expo^{\imag \skp{y}{p}} f(p)
	\\
	&=
	\int \frac{\ddx p}{(2\pi)^4}\int_{\frac{\delta}{\imag\pp}}^{1+\frac{\delta}{\imag\pp}} \dx \overline u \frac{1-\overline u-0}{ \overline u } \expo^{\imag \skp{y}{p}} f(p)
	\\
	&=
	\int \frac{\ddx p}{(2\pi)^4}\left(-1+\ln\left(\frac{\imag\pp}{\delta}\right)\right) \expo^{\imag \skp{y}{p}} f(p)
\end{align}
note that i have in my previous computation a different integration order, this minus sign.
From the two cases $y=\pm zn$ there are two results
\begin{align}
	\int_{-\infty}^{0} \dx \sigma \int_0^1 \dx u u \partial_+ f(\pm zn+\overline u \sigma n)	
	&=
	-\left(-1+\ln\left(\frac{\pm\imag \pphat}{\delta}\right)\right) f(\pm zn)
\end{align}



then the final result is
\begin{align}
	\frac{  g^2 C_A}{4\pi^2 \epsilon} \left(-1+\ln\left(\frac{\pm\imag \pphat}{\delta}\right)\right)\ft^C_{\mu +}(\pm zn)
\end{align}



Other diagram.

\begin{align}
&
gf^{A'B'C'} B_\mu^{B'}(zn)B_+^{C'}(zn) \idm \left(-\imag g \int \ddx x A_\nu ^D (x) \partial_\alpha B_\beta^E(x) B_\gamma^F(x) \right) v^{\nu\alpha\beta\gamma}_{DEF}
\\
&\rightarrow
\contraction[2ex]{gf^{A'B'C'} }{B}{_\mu^{B'}(zn)B_+^{C'}(zn) \idm \big(-\imag g \int \ddx xA_\nu ^D (x) \partial_\alpha}{B}
\contraction[3ex]{gf^{A'B'C'} B_\mu^{B'}(zn)}{B}{_+^{C'}(zn) \idm \big(-\imag g \int \ddx xA_\nu ^D (x) \partial_\alpha B_\beta^E }{B}
{\color{blue}gf^{A'B'C'} B_\mu^{B'}(zn)B_+^{C'}(zn) \idm \big(-\imag g \int \ddx xA_\nu ^D (x) \partial_\alpha B_\beta^E B_\gamma^F(x) \big) v^{\nu\alpha\beta\gamma}_{DEF}}
\nl
+
\contraction[3ex]{gf^{A'B'C'} }{B}{\mu^{B'}(zn)B_+^{C'}(zn) \idm \big(-\imag g \int \ddx xA_\nu ^D (x) \partial_\alpha B_\beta^E}{B}
\contraction[2ex]{gf^{A'B'C'} B_\mu^{B'}(zn)}{B}{_+^{C'}(zn) \idm \big(-\imag g \int \ddx xA_\nu ^D (x) \partial_\alpha }{B}
{\color{red}
gf^{A'B'C'} B_\mu^{B'}(zn)B_+^{C'}(zn) \idm \big(-\imag g \int \ddx xA_\nu ^D (x) \partial_\alpha B_\beta^E B_\gamma^F(x) \big) v^{\nu\alpha\beta\gamma}_{DEF}}
\end{align}
here i compute both terms/diagrams, but again split common factor. 
color computation:
\begin{align}
	&f^{A'B'C'}f^{DEF} ({\color{blue}\delta^{B'E}\delta^{C'F}} +{\color{red}\delta^{B'F}\delta^{C'E}})
	\\
	&=
	f^{A'EF}f^{DEF} ({\color{blue}+1} +{(\color{red}-1)})
	\\
	&=
	C_A \delta^{A'D} ({\color{blue}+1} +{(\color{red}-1)})
\end{align}
\begin{align}
	-\imag g^2 \int \ddx x A_\nu(x) 	C_A \delta^{A'D}v^{\nu\alpha\beta\gamma}
	\\
	\partial_{x^\alpha}\propagator_{\mu \beta}(zn-x) \propagator_{\gamma +} (zn-x)
	-
	\propagator_{\mu \gamma}(zn-x) \partial_{x^\alpha}\propagator_{\beta +}(zn-x)
\end{align}

\begin{align}
	\frac{-\imag g^2 \Gamma^2\left( \frac{d}{2}-1\right)}{4^2\pi^d}\int \ddx x A^{A'}_\nu(x)	C_A  v^{\nu\alpha\beta\gamma}
	\\
	&\frac{-2(\frac{d}{2}-1)(x-zn)_\alpha}{\left(-(x-zn)^2+\imag \epsilon\right)^{\frac{d}{2}}}\frac{1}{\left(-(x-zn)^2+\imag \epsilon\right)^{\frac{d}{2}}-1}(\metric_{\gamma +}\metric_{\mu \beta}
	-\metric_{\beta +}\metric_{\mu \gamma})
	\\
	&=
	\int_0^1 \dx u u^{\frac{d}{2}-1}\overline u^{\frac{d}{2}-2} \frac{\Gamma(d-1)}{\Gamma\left(\frac{d}{2}-1\right)\Gamma\left(\frac{d}{2}\right)}\frac{-2(\frac{d}{2}-1)(x-zn)_\alpha}{\left(-(x-zn)^2+\imag \epsilon\right)^{d-1}}(\metric_{\gamma +}\metric_{\mu \beta}
	-\metric_{\beta +}\metric_{\mu \gamma})
	\\
	&=
	\int_0^1 \dx u u^{\frac{d}{2}-1}\overline u^{\frac{d}{2}-2} \frac{\Gamma(d-1)}{\Gamma^2\left(\frac{d}{2}-1\right)}\frac{-2(x-zn)_\alpha}{\left(-(x-zn)^2+\imag \epsilon\right)^{d-1}}(\metric_{\gamma +}\metric_{\mu \beta}
	-\metric_{\beta +}\metric_{\mu \gamma})
\end{align}
The shift is $x\rightarrow x+zn$.
The metric can be contracted:
\begin{align}
	(\metric_{\gamma +}\metric_{\mu \beta}
	-\metric_{\beta +}\metric_{\mu \gamma})v^{\nu\alpha\beta\gamma}
	&=
	4(\metric_\mu^\nu\metric^{\alpha+}-\metric_\mu^\alpha \metric_{\nu+})
\end{align}
\begin{align}
\frac{-\imag g^2 C_A\Gamma\left( d-1\right)}{4^2\pi^d} \int_0^1 \dx u u^{\frac{d}{2}-1}\overline u^{\frac{d}{2}-2}4(\metric_\mu^\nu\metric^{\alpha+}-\metric_\mu^\alpha \metric_{\nu+})
\\
&\int \ddx x\frac{-2(x-zn)_\alpha x_\tau}{\left(-(x)^2+\imag \epsilon\right)^{d-1}}\partial^\tau A^{A'}_\nu(zn)
\\
&=
\frac{+2\imag \pi^{\frac{d}{2}}}{2\epsilon} \frac{1}{\Gamma(d-1)} \partial_\alpha A^{A'}_\nu(zn)
\end{align}
Have in total

\begin{align}
	\frac{ g^2 C_A}{4\pi^2 \epsilon} 
	(\partial_+A_\mu-\partial_\mu A_+)^{A'}(zn)
	\\
	&=
	-\frac{ g^2 C_A}{4\pi^2 \epsilon} 
	\ft_{\mu +}^{A'}(zn)
\end{align}
Looks ok but not sure about factor. So this has to be checked again.  But it does not contribute to the pole with regulator $\delta$.
CHECKED IT, CORRECTED IT AND I THINK FACTOR IS OK NOW. \\
For other diagram, meaning other side of wilson line there is NO difference in color and also nothing else that makes me think there would be a difference. So it is the same result (of course at other space-time point.) 

Next diagram:\\
\begin{align}
	&
	gf^{A'B'C'} B_\mu^{B'}(zn)B_+^{C'}(zn) \int_{-\infty} ^{z} \dx \sigma \imag g B_+(\sigma n) 
\end{align}
To the outgoing field has to be classical, so I have to choose which one it should be. However Once needs to understand that already here the suspicion that the diagram is null can be made. Two contractions:
\begin{align}
&
\contraction[2ex]{gf^{A'B'C'} }{B}{_\mu^{B'}(zn)A_+^{C'}(zn) \int_{-\infty} ^{z} \dx \sigma \imag g}{B}
gf^{A'B'C'} B_\mu^{B'}(zn)A_+^{C'}(zn) \int_{-\infty} ^{z} \dx \sigma \imag g B_+^D(\sigma n)T^D_{A'C}
\nl
\contraction[2ex]{+gf^{A'B'C'} A_\mu^{B'}(zn)}{B}{_+^{C'}(zn) \int_{-\infty} ^{z} \dx \sigma \imag g }{B}
+gf^{A'B'C'} A_\mu^{B'}(zn)B_+^{C'}(zn) \int_{-\infty} ^{z} \dx \sigma \imag g B_+^D(\sigma n)T^D_{A'C} 
\end{align}
color is of course a relative minus. But the second diagram is null since $n^2$ is 0. First diagram should be null in light cone gauge, and because $g_{\mu +} $ is 0. So anyway it is 0.

Now the results:
The diagrams are in talbe \tref{tablewithdiagramsforgluontwist1}:

\begin{table}
	\begin{tabular}{l l l}
		
		\def \thelable {A}
		\begin{tikzpicture}
		\begin{feynman}
		\vertex(field);
		\vertex[right=1.0cm of field](wilsoncoupling);
		\vertex[right=2.0cm of wilsoncoupling](infinity);
		\vertex[below=1cm of field](3gluon);
		\vertex[below=1cm of 3gluon](outgoingfield);
		\diagram* {(field) -- [scalar] (wilsoncoupling) -- [scalar] (infinity)};
		\diagram* {(field) -- [boson] (3gluon) -- [boson] (wilsoncoupling)};
		\diagram* {(3gluon) -- [boson] (outgoingfield)};
		\vertex[right=2.0cm of field](auxpoint);
		\vertex[below=1.5cm of auxpoint](label){\(\thelable\)};
		\end{feynman}
		\end{tikzpicture}
		
		&
		\def \thelable {B}
		\begin{tikzpicture}
		\begin{feynman}
		\vertex(field);
		\vertex[right=1.0cm of field](wilsoncoupling);
		\vertex[right=2.0cm of wilsoncoupling](infinity);
		\vertex[below=1cm of field](3gluon);
		\vertex[below=1cm of 3gluon](outgoingfield);
		\diagram* {(field) -- [scalar] (wilsoncoupling) -- [scalar] (infinity)};
		\diagram* {(field) -- [boson,half left] (3gluon) -- [boson,half left] (field)};
		\diagram* {(3gluon) -- [boson] (outgoingfield)};
		\vertex[right=2.0cm of field](auxpoint);
		\vertex[below=1.5cm of auxpoint](label){\(\thelable\)};
		\end{feynman}
		\end{tikzpicture}
		
		&
		\def \thelable {C}
		\begin{tikzpicture}
		\begin{feynman}
		\vertex(field);
		\vertex[right=1.0cm of field](wilsoncoupling);
		\vertex[right=2.0cm of wilsoncoupling](infinity);
		\vertex[below=1cm of field](3gluon);
		\vertex[below=1cm of 3gluon](outgoingfield);
		\diagram* {(field) -- [scalar] (wilsoncoupling) -- [scalar] (infinity)};
		\diagram* {(field) -- [boson,half left] (wilsoncoupling)};
		\diagram* {(field) -- [boson] (outgoingfield)};
		\vertex[right=2.0cm of field](auxpoint);
		\vertex[below=1.5cm of auxpoint](label){\(\thelable\)};
		\end{feynman}
		\end{tikzpicture}
	\end{tabular}
	\caption{diagrams for gluon TMD opertator on twist 1}
	\label{tablewithdiagramsforgluontwist1}
\end{table}
The results are
\begin{align}
	A&=
	\frac{  g^2 C_A}{4\pi^2 \epsilon} \left(-1+\ln\left(\frac{\imag \pphat}{\delta}\right)\right)\ft^C_{\mu +}(zn)
	\\
	B&=
	-\frac{ g^2 C_A}{4\pi^2 \epsilon} 
	\ft_{\mu +}^{A'}(zn)
	\\
	C&=
	0
\end{align}

\ifdefined\mainprogram{}
\else
\end{document}

\fi
