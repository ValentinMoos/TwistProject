\ifdefined\mainprogram{}
\else
\documentclass[10pt]{article}

%packages +  bib
\usepackage[left=2.5cm,right=2.5cm,bottom=2cm,top=3.0cm]{geometry}
\usepackage{amsmath}      %std package for many operators
\usepackage{amssymb}      %symbolds i guess
\usepackage{bbold}	  %identity matrix
\usepackage{ dsfont }	  % i have no idea
\usepackage{color}              % for comments
\usepackage{xcolor}	   %for additional color, can be deleted in the end i think!!!s
\usepackage{physics}          % bra ket notation
\usepackage{slashed}          %for slash notation
\usepackage{fancyhdr}       %allows chance of layout
\usepackage[subfigure]{tocloft}    %TOC layout
\usepackage{indentfirst}      %for parindent after new section
\usepackage{graphicx,subfigure}	  %for uni logo
\usepackage{setspace}      %for spacing between the lines
\usepackage{simplewick}   %for wick contractions
\usepackage[compat=1.1.0]{tikz-feynman} %SM diagrams
\usepackage[title,titletoc]{appendix} %appendix written in table of contents
\usepackage[sorting=none]{biblatex}  %numbers of citation are how they appear in tex
\bibliography{literature.bib}

%Layout including how to enumberate equations
\setlength\parindent{0.8cm}
\setlength{\cftsecnumwidth}{35pt}
\setlength{\cftsubsecnumwidth}{35pt}
\pagestyle{fancy} %benutzerdef
\fancyhf{}
\fancyhead[R]{\thepage}
\renewcommand{\headrulewidth}{0pt}

\renewcommand{\thesection}{\Roman{section}} % i removed the ``.'' in the end ( in subsection it is still there) for references. It should work this way.
\renewcommand{\thesubsection}{\Alph{subsection}}
\renewcommand{\thesubsubsection}{\Alph{subsection}.\roman{subsubsection}}



\numberwithin{equation}{section}
\renewcommand{\theequation}{\arabic{section}.\arabic{equation}}

\renewcommand{\baselinestretch}{1.25}

%Additional Layout matters
\newcommand{\remark}[1]{\newline\newline \emph{#1} \newline\newline}

%dates


%Other / Names
\newcommand{\Mat}{\text{\emph{Mathematica}}}


%References
\newcommand{\howtowriteequation}{eq.$~$}
\newcommand{\cref}[1]{eq.$~$(#1)} % cite equation from other paper ref
\newcommand{\eref}[1]{\howtowriteequation (\ref{#1})}
\newcommand{\doubleref}[2]{\howtowriteequation (\ref{#1}, \ref{#2})}
\newcommand{\tripleref}[3]{\howtowriteequation (\ref{#1}, \ref{#2}, \ref{#3})}
\newcommand{\ereffromto}[2]{\howtowriteequation (\ref{#1})-(\ref{#2})}

\newcommand{\sref}[1]{section$~$\ref{#1}}
\newcommand{\Sref}[1]{Section$~$\ref{#1}}
\newcommand{\tref}[1]{table$~$\ref{#1}}
\newcommand{\fref}[1]{figure$~$\ref{#1}}
\newcommand{\aref}[1]{appendix$~$\ref{#1}}

\newcommand{\pref}[1]{page$~$\pageref{#1}}

\newcommand{\mc}[1]{\cite{#1}}

%Notes for correcting
\newcommand{\COM}[1]{\text{\textcolor{red}{#1}}}
\newcommand{\LBL}[0]{\COM{Hier}}
\newcommand{\spacee}{~~~~~~~}
%newline in formula with term exeeding length of a line
\newcommand{\nl}{\\&~~~}

%Functions and symbols
\newcommand{\gf}{G}
\newcommand{\hamiltonian}{H}
\newcommand{\flnso}{Op.}
\newcommand{\propagator}{\Delta}  % propagator symbol

%Fields (quark, gauge etc)
\newcommand{\Aqu}{B} %quantum gauge field
%operators
\newcommand{\operatorO}{O}
\newcommand{\twoq}[1]{\operatorO_q^{#1}}
\newcommand{\twoqbar}[1]{\overline \operatorO_{q}^{#1}}
\newcommand{\twog}[1]{\operatorO_{g}^{#1}}
\newcommand{\twogbar}[1]{\overline \operatorO_{g}^{#1}}

%Differential Operators
\newcommand{\dx}{\text{d}}
\newcommand{\abdif}[1]{\frac{\dx}{\dx #1  } }
\newcommand{\abdifof}[2]{\frac{\dx #1}{\dx #2  } }
\newcommand{\padif}[1]{\frac{\partial}{\partial #1  } }
\newcommand{\doublepadif}[2]{\frac{\partial^2}{\partial #1 \partial #2}}
\newcommand{\padifntimes}[2]{\frac{\partial^{#2}}{\partial #1^{#2}}}
\newcommand{\ft}{F}%field tensor
\newcommand{\cov} {D}
\newcommand{\covleft}{\overleftarrow{\cov}}
\newcommand{\covright}{\overrightarrow{\cov}}
\newcommand{\dxabc}{\left[\dx \alpha \dx \beta \dx \gamma\right]}
\newcommand{\dxabcprime}{\left[\dx \alpha' \dx \beta' \dx \gamma'\right]}
\newcommand{\dxfm}[2]{\frac{\dx^{#2}#1}{\left(2\pi\right)^{#2}}}
\newcommand{\dxfs}[2]{\dx^{#2}#1}
\newcommand{\ddx}{\dx ^d}


%Michelaneous including spinors
\newcommand{\metric}{\eta}
\newcommand{\idm}{{1}}
\newcommand{\abelian}[2]{\left[ #1, #2 \right]}
\newcommand{\expo}{e}
\newcommand{\skp}[2]{(#1|#2)}
\newcommand{\skpt}[2]{(#1\cdot#2)}
\newcommand{\spp}[1]{\langle #1 \rangle}
\newcommand{\aspp}[1]{\left[ #1 \right]}
\renewcommand{\trace}[0]{\text{Tr}}

%spinors
\newcommand{\psibar}{\overline\psi}
\newcommand{\psibarc}{\psibar_c}
\newcommand{\psibarq}{\psibar_q}
\newcommand{\psic}{\psi_c}
\newcommand{\psiq}{\psi_q}
\newcommand{\psip}{\psi_{+}}
\newcommand{\psim}{\psi_{-}}
\newcommand{\psit}{\overline\psi}
\newcommand{\chit}{\overline\chi}
\newcommand{\psipt}{\psit_{+}}
\newcommand{\psimt}{\psit_{-}}
\newcommand{\chip}{\chi_{+}}
\newcommand{\chim}{\chi_{-}}
\newcommand{\chipt}{\chit_{+}}
\newcommand{\chimt}{\chit_{-}}
\newcommand{\ispace}{~\hspace{-3pt}}
\newcommand{\ospex}{O}
\newcommand{\spideriv}[2]{\frac{#1\partial}{\partial #2}}
\newcommand{\spiderivb}[3]{\left(\spideriv{#1}{#2}\right)^{#3}}
\newcommand{\fbar}{\overline f}
\newcommand{\SX}{J}

%%Spinor Combinations
\newcommand{\lol}{\lambda\overline\lambda}
\newcommand{\mom}{\mu\overline\mu}
\newcommand{\lom}{\lambda\overline\mu}
\newcommand{\mol}{\mu\overline\lambda}
\newcommand{\olom}{\overline\lambda\overline\mu}
\newcommand{\omol}{\overline\mu\overline\lambda}
\newcommand{\om}{\overline\mu}
\newcommand{\ol}{\overline\lambda}
%skp that appear quite often such that the order is consistent
\newcommand{\skpyP}{\skp{y}{P}}
\newcommand{\skpPy}{\skpyP}

\newcommand{\skpyS}{\skp{y}{S}}
\newcommand{\skpSy}{\skpyS}
% q Bar, bold B , pplus, nbar
\newcommand{\qbar}{\overline q}
\newcommand{\bb}{\boldsymbol{b}}
\newcommand{\pp}{p_{+}}
\newcommand{\pphat}{\hat p_{+}}
\newcommand{\nbar}{\tilde n}
%Imaginary Unity, coupling: 
\newcommand{\imag}{\text{i}}
\newcommand{\as}{\alpha_s}
%Operators
%Wilson line, pretzelosity , Matrix element PS
\newcommand{\brez}{h_{1T}^{\bot}}
\newcommand{\wil}[2]{\left[#1,#2 \right]}
\newcommand{\myState}{P,S}
\newcommand{\fme}[1]{\bra{\myState}#1\ket{\myState}}
\newcommand{\twist}{T}
%O Gamma operators
\newcommand{\oga}{O ^{\Gamma  }}
\newcommand{\ogmu}{O^{\gamma^\mu}}
\newcommand{\ogmugfive}{O^{\gamma^\mu \gamma^5}}
\newcommand{\osigma}{O^{\imag \sigma ^{\mu \nu }\gamma_5}}
\newcommand{\ogplus}{O^{\gamma^{+}}}
\newcommand{\ogplusgfive}{O^{\gamma^{+} \gamma^5}}
\newcommand{\osigmaplus}{O^{\imag \sigma ^{\alpha + }\gamma_5}}
\newcommand{\osigmavarplus}[1]{O^{\imag \sigma ^{#1 + }\gamma_5}}
%3point operators
\newcommand{\ttp}{\mathcal{T}^\Gamma}
\newcommand{\ttpg}{\mathcal{T}^{\gamma_+}}
\newcommand{\ttpgg}{\mathcal{T}^{\gamma_+\gamma_5}}
\newcommand{\ttpsg}{\mathcal{T}^{\imag \sigma^{\alpha + }\gamma^5}}
\newcommand{\ttpsgadaptive}[3]{\mathcal{T}^{\imag \sigma^{#1 #2}\gamma^5}_{#3}}
%matrix elements of 3 point operators
\newcommand{\Dt}{\Delta T}
\newcommand{\Dtx}{\Dt (x_1,x_2,x_3)}
\newcommand{\Dtt}{\Delta \tilde T}
\newcommand{\dttg}{\delta \tilde T_g}
\newcommand{\dtte}{\delta \tilde T_\expsilon}
\newcommand{\dTe}{\delta T _\epsilon}
\newcommand{\dTex}{\delta T _\epsilon(x_1,x_2,x_3)}
\newcommand{\dTg}{\delta T _g}
\newcommand{\dTgx}{\delta T _g(x_1,x_2,x_3)}
\newcommand{\xonetwothree}{x_{1,2,3}}
%U gamma operators or full TMD s
\newcommand{\uga}{\mathcal{U}^{\Gamma}}
\newcommand{\udis}{\mathcal{U}_{DIS}^{\Gamma}}
\newcommand{\udy}{\mathcal{U}_{DY}^{\Gamma}}
\newcommand{\udygplus}{\mathcal{U}_{DY}^{\gamma^+}}
\newcommand{\udygplusgfive}{\mathcal{U}_{DY}^{\gamma^+\gamma_5}}
\newcommand{\udysg}{\mathcal{U}_{DY}^{\imag \sigma^{\alpha +}_{T} \gamma_5}}

\newcommand{\nameofO}{O}	
\newcommand{\ogt}{\nameofO_{TMD}^{\Gamma}}
\newcommand{\ogtfields}[2]{\nameofO_{#1}^{#2}}
\newcommand{\ogpt}{\nameofO_{TMD}^{\gamma_+}}

%phi distributions
\newcommand{\phiqh}{\Phi_{q\leftarrow h}}
\newcommand{\phiqhij}{\Phi_{q\leftarrow h,ij}}
\newcommand{\phiG}{\Phi_{q\leftarrow h}^{[\Gamma]}}
\newcommand{\phig}{\Phi_{q\leftarrow h}^{[\gamma^+]}}
\newcommand{\phigg}{\Phi_{q\leftarrow h}^{[\gamma^+ \gamma_5]}}
\newcommand{\phisg}[1]{\Phi_{q\leftarrow h}^{[\sigma^{#1 +}\gamma_5]}}

%Parametrization fucntions
\newcommand{\paraA}{A}
\newcommand{\paraB}{B}


%Graphics

\usepackage[]{hyperref}

\begin{document}

\fi

1-loop diagrams for gluon pdf: twist2:

operators with this twist should look have the form
\begin{align}
	\int \dx \sigma \ft_{\mu+} \wil{zn}{\sigma n} \ft_{\nu +} \wil{\sigma }{-\infty n}
\end{align}
Problably it is ok to ignore the integraion which should be kept until the end anyway? 
Like i have learned before the color of the whole object is connected, and just one index is open here. 
so I need to know what to do for: 

no disturbance
disturbance in first path
disturbance in second path
\begin{align}
	&\ft_{\mu+}^{A}  \ft_{\nu +}^{B} T^{B}_{AC}
	\\
	&\imag g \int_{\sigma}^{z} \dx \tau \ft_{\mu+}^{A} B_+^B(\tau n) T^B_{AC}  \ft_{\nu +}^{D}T^D_{CE}
	\\
	&\imag g \int_{-\infty}^{\sigma} \dx \tau \ft_{\mu+}^{A} \ft_{\nu +}^{B}T^{B}_{AC} B_+^D(\tau n) T^D_{CE}
\end{align}
Note: I will compute for antisymmetric part of tensor: (i know it is fully as) $partial_\mu A_M$, since $M$ is capital $\mu$. at some convenient point in computation i will subract interchanged term and replace $M$ by $+$.  Same for $\nu \leftrightarrow N$.
with this:

Diagram A:
\begin{align}
	&\ft_{\mu M}^{A} \imag g B_+^B  T^B_{AC} \ft _{\nu N}^D T^D_{CE}\bigg(-\imag g \int \ddx xA_\iota^{A'}(x) \partial_{x^\alpha} B_\beta^{B'} (x) B_\gamma^{C'}(x) \bigg) v^{\iota \alpha\beta\gamma}_{A'B'C'}
	\\
	&\rightarrow
	\\
	\contraction[3ex]{}{\ft}{_{\mu M}^{A} \imag g B_+^B  T^B_{AC} \ft _{\nu N}^D T^D_{CE}\bigg(-\imag g \int \ddx xA_\iota^{A'}(x) \partial_{x^\alpha}}{B}
	\contraction[2ex]{\ft_{\mu M}^{A} \imag g}{B}{_+^B  T^B_{AC} \ft _{\nu N}^D T^D_{CE}\bigg(-\imag g \int \ddx xA_\iota^{A'}(x) \partial_{x^\alpha} B_\beta^{B'} (x)}{B}
	&\ft_{\mu M}^{A} \imag g B_+^B  T^B_{AC} \ft _{\nu N}^D T^D_{CE}\bigg(-\imag g \int \ddx xA_\iota^{A'}(x) \partial_{x^\alpha} B_\beta^{B'} (x) B_\gamma^{C'}(x) \bigg) v^{\iota \alpha\beta\gamma}_{A'B'C'}
	\nl
	+
	\contraction[3ex]{}{\ft}{_{\mu M}^{A} \imag g B_+^B  T^B_{AC} \ft _{\nu N}^D T^D_{CE}\bigg(-\imag g \int \ddx xA_\iota^{A'}(x) \partial_{x^\alpha} B_\beta^{B'} (x)}{B}
	\contraction[2ex]{\ft_{\mu M}^{A} \imag g}{B}{_+^B  T^B_{AC} \ft _{\nu N}^D T^D_{CE}\bigg(-\imag g \int \ddx xA_\iota^{A'}(x) \partial_{x^\alpha}}{B}
	\ft_{\mu M}^{A} \imag g B_+^B  T^B_{AC} \ft _{\nu N}^D T^D_{CE}\bigg(-\imag g \int \ddx xA_\iota^{A'}(x) \partial_{x^\alpha} B_\beta^{B'} (x) B_\gamma^{C'}(x) \bigg) v^{\iota \alpha\beta\gamma}_{A'B'C'}	
\end{align}

I claim that the computation is - after computing color - 100\% identical to the computation for twist 1. The intermediate field is simply ignored. (one of course has to add the diagram $A'$).
color is 
\begin{align}
	(\delta^{AB'} \delta^{BC'} -\delta^{AC'}\delta^{BB'}) f^{A'B'C'} T^{B}_{AC} T^{D}_{CE}
	&= - \imag f^{A'AB}f^{BAC} T^D_{CE}(1-1)
	\\
	&=
	\imag C_A T_{A'E}^D
\end{align}
so if i trust my intuition i end with the expression
\begin{align}
	\pm\int_{-\infty}^{\pm z} \dx \sigma\frac{  g^2 C_A}{4\pi^2 \epsilon} \left(-1+\ln\left(\frac{\pm\imag \pphat}{\delta}\right)\right)
	\ft_{\mu +}^{A'}(\pm zn)\ft_{\nu +}^D(\sigma n)T^D_{A'E}
\end{align}
all $+$-es are for field at $zn$, and all $-$-es are for other case. Minuses i am not so sure, should check. Most likely there is a sign flip in this expression. 
I need to check it.

Diagram B:
here i show full computation.
we start with the same expression as for diagram A. 
\begin{align}
	&\ft_{\mu M}^{A} \imag g B_+^B  T^B_{AC} \ft _{\nu N}^D T^D_{CE}\bigg(-\imag g \int \ddx xA_\iota^{A'}(x) \partial_{x^\alpha} B_\beta^{B'} (x) B_\gamma^{C'}(x) \bigg) v^{\iota \alpha\beta\gamma}_{A'B'C'}
	\\
	&\rightarrow
	\\
	&=
	\contraction[2ex]{\ft_{\mu M}^{A} \imag g B_+^B  T^B_{AC} }{\ft}{_{\nu N}^D T^D_{CE}\bigg(-\imag g \int \ddx xA_\iota^{A'}(x) \partial_{x^\alpha}}{B}
	\contraction[3ex]{\ft_{\mu M}^{A} \imag g}{B}{_+^B  T^B_{AC} \ft _{\nu N}^D T^D_{CE}\bigg(-\imag g \int \ddx xA_\iota^{A'}(x) \partial_{x^\alpha} B_\beta^{B'} (x)}{B}
	\ft_{\mu M}^{A} \imag g B_+^B  T^B_{AC} \ft _{\nu N}^D T^D_{CE}\bigg(-\imag g \int \ddx xA_\iota^{A'}(x) \partial_{x^\alpha} B_\beta^{B'} (x) B_\gamma^{C'}(x) \bigg) v^{\iota \alpha\beta\gamma}_{A'B'C'}
	\nl
	\contraction[3ex]{+\ft_{\mu M}^{A} \imag g }{B}{_+^B  T^B_{AC} \ft _{\nu N}^D T^D_{CE}\bigg(-\imag g \int \ddx xA_\iota^{A'}(x) \partial_{x^\alpha}}{B}
	\contraction[2ex]{+\ft_{\mu M}^{A} \imag g B_+^B  T^B_{AC}}{\ft}{_{\nu N}^D T^D_{CE}\bigg(-\imag g \int \ddx xA_\iota^{A'}(x) \partial_{x^\alpha} B_\beta^{B'} (x)}{B}
	+\ft_{\mu M}^{A} \imag g B_+^B  T^B_{AC} \ft _{\nu N}^D T^D_{CE}\bigg(-\imag g \int \ddx xA_\iota^{A'}(x) \partial_{x^\alpha} B_\beta^{B'} (x) B_\gamma^{C'}(x) \bigg) v^{\iota \alpha\beta\gamma}_{A'B'C'}
\end{align}
unfortunately i forgot to put arguments, on the fields of the operator, but they are clear.

the color factor requires the identity for 3 structure function product. 
it can de easily derived from Jacobi-identity (for structure functions) plus identity for product of two structure functions.
the formula reads
\begin{align}
	f^{DEA}f^{EFB}f^{FDC} &= \frac{1}{2}C_A f^{ABC}
\end{align}
The color factor compuation for this diagrams is
\begin{align}
	 T^{B}_{AC} T^{D}_{CE} f^{A'B'C'}( \delta^{BC}\delta^{DB'} + \delta^{BB'}\delta^{DC'})
	&=
	-(-\imag)^2f^{CBA}f^{BDA'}f^{DCE} (1-1)
	\\
	&=
	\frac{- C_A}{2}(-\imag)^2f^{AA'E} (1-1)
\end{align}
again the minus is between different possibilities of contracting the fields. 
We can remove the remaining structure function back to the color matrix to restore the initial expression.
\begin{align}
 T^{B}_{AC} T^{D}_{CE} f^{A'B'C'}( \delta^{BC}\delta^{DB'} + \delta^{BB'}\delta^{DC'})
	&=
	\frac{-\imag C_A}{2} T^{A'}_{AE} (1-1)
\end{align}


The terms yield

\begin{align}
	&-\imag^2 g^2 \ft_{\mu M}^{A} \int \ddx  A_\iota^{A'}(x)\frac{-\imag C_A}{2} T^{A'}_{AE}v^{\iota \alpha\beta\gamma}
	\\
	&\propagator_{+\gamma}(\tau n- x) \partial_{\nu} \partial_{\alpha}\propagator_{N\beta}(\sigma n -x) - \partial_{\alpha}\propagator_{+ \beta}(\tau n - x)\partial_{\nu}\propagator_{N \gamma}(\sigma n-x)
\end{align}

always keep in mind to which argument the derivatives are related. I do omit the extra term arising from double derivative on the propagator, since its contribution is null (Lorentz contracted is 0)

\begin{align}
	&-\imag^2 g^2 \ft_{\mu M}^{A} \int \ddx  A_\iota^{A'}(x)\frac{-\imag C_A}{2} T^{A'}_{AE}v^{\iota \alpha\beta\gamma}\Gamma^2\left(\frac{d}{2}-1\right)\frac{1}{4^2\pi^\frac{d}{2}}
	\\
	&\frac{1}{\left(-(\tau n -x)^2+\imag \epsilon\right)^{\frac{d}{2}-1}}\frac{4\left(\frac{d}{2}-1\right)\frac{d}{2}(\sigma n -x)_\nu(x-\sigma n)_\alpha}{\left(-(\sigma n -x)^2+\imag \epsilon\right)^{\frac{d}{2}+1}}\metric_{\gamma +}\metric_{N \beta}
	\nl
	-\frac{-2\left(\frac{d}{2}-1\right)(x -\tau n)_\alpha}{\left(-(\tau n -x)^2+\imag \epsilon\right)^{\frac{d}{2}}}\frac{-2\left(\frac{d}{2}-1\right)( \sigma n -x )_\nu}{\left(-(\sigma n -x)^2+\imag \epsilon\right)^{\frac{d}{2}}}\Gamma(d) \metric_{\beta +}\metric_{N \gamma}
\end{align}

\begin{align}
	&-\imag^2 g^2 \ft_{\mu M}^{A} \int \ddx  A_\iota^{A'}(x)\frac{-\imag C_A}{2} T^{A'}_{AE}v^{\iota \alpha\beta\gamma}\Gamma^2\left(\frac{d}{2}-1\right)\frac{1}{4^2\pi^\frac{d}{2}}4 \frac{1}{\Gamma^2\left(\frac{d}{2}-1\right)}
	\\
	&\int_0^1 \dx u u^{\frac{d}{2}-2}\overline u^{\frac{d}{2}}\frac{-x_\nu(x-\sigma n)_\alpha}{\left(-u(\tau n -x)^2-\overline u(\sigma n -x)^2+\imag \epsilon\right)^{d}}\metric_{\gamma +}\metric_{N \beta}
	\nl
	+\int_0^1 \dx u u^{\frac{d}{2}-1}\overline u^{\frac{d}{2}-1}\frac{x_\nu(x- \tau  n )_\alpha}{\left(-u(\tau n -x)^2-\overline u(\sigma n -x)^2+\imag \epsilon\right)^{d}}\metric_{\beta +}\metric_{N \gamma}
\end{align}

The shifts in this setup are:
\begin{align}
	\tau \rightarrow &\tau + \sigma \text{   first shift}
	\\
	x \rightarrow & x + u\tau n + \sigma n
	\\
	x-\sigma n \rightarrow & x + u \tau n
	\\
	x-\tau n \rightarrow & x  - \overline u \tau n 
\end{align}
compute metric
\begin{align}
	\metric_{\gamma +}\metric_{N \beta}v_{\iota \alpha \beta\gamma} 
	&=
	2(\metric_{\iota N}\metric_{\alpha +} - \metric_{\iota+ }\metric_{\alpha N})
	\\
	&=
	-\metric_{\beta +}\metric_{N \gamma}v_{\iota \alpha \beta\gamma} 
\end{align}
Hence another relative sign between term and identical metric term
Have:
\begin{align}
	&-\imag^2 g^2 \ft_{\mu M}^{A} \int \ddx  A_\iota^{A'}(x)\frac{-\imag C_A}{2} T^{A'}_{AE}\Gamma^2\left(\frac{d}{2}-1\right)\frac{1}{4^2\pi^\frac{d}{2}}4 \frac{1}{\Gamma^2\left(\frac{d}{2}-1\right)}\int_0^1 \dx u u^{\frac{d}{2}-2}\overline u^{\frac{d}{2}-1}2(\metric_{\iota N}\metric_{\alpha +} - \metric_{\iota+ }\metric_{\alpha N})
	\\
	&\frac{-\overline u x_\nu(x+u \tau  n)_\alpha- u x_\nu(x - \overline u \tau n )_\alpha }{\left(-x^2+\imag \epsilon\right)^{d}} =\frac{-x_\nu x_\alpha}{\left(-x^2+\imag \epsilon\right)^{d}}
\end{align}
Then from integration we have doulbe derivative

\begin{align}
	\int \ddx x \frac{-x_\nu x_\alpha x_\rho x_\kappa}{\left(-x^2+\imag \epsilon\right)^{d}} \partial_\rho \partial_\kappa A^{A'}(u\tau n + \sigma n)
	&=
	\frac{-\imag \pi^{\frac{d}{2}}}{4(2-\frac{d}{2})\Gamma(d)}\left(  \metric_{\nu \alpha} \metric_{\rho \kappa}+ \metric_{\nu \rho} \metric_{\alpha\kappa} + \metric_{\nu \kappa} \metric_{\alpha \rho} \right) \partial^\kappa\partial^\rho A_\iota^{A'} (\sigma n+u\tau n)
\end{align}
now again contract metrics: 

\begin{align}
	\left(  \metric_{\nu \alpha} \metric_{\rho \kappa}+ \metric_{\nu \rho} \metric_{\alpha\kappa} + \metric_{\nu \kappa} \metric_{\alpha \rho} \right) \partial^\kappa\partial^\rho A_\iota^{A'} (\sigma n+u\tau n)(\metric_{\iota N}\metric_{\alpha +} - \metric_{\iota+ }\metric_{\alpha N})
	&=
	2 \partial_+\left( \partial_\nu A_+ - \partial_+A_\nu\right)^{A'}(\sigma n + u \tau n)
\end{align}
Total , combination of last two steps
\begin{align}
	&-\imag^2 g^2 \ft_{\mu M}^{A}\frac{-\imag C_A}{2} T^{A'}_{AE}\Gamma^2\left(\frac{d}{2}-1\right)\frac{1}{4^2\pi^\frac{d}{2}}4 \frac{1}{\Gamma^2\left(\frac{d}{2}-1\right)}
	\\
	&\int_0^1 \dx u u^{\frac{d}{2}-2}\overline u^{\frac{d}{2}-1}2~2 \partial_+\left( \partial_\nu A_+ - \partial_+A_\nu\right)^{A'}(\sigma n + u \tau n)\frac{-\imag \pi^{\frac{d}{2}}}{4(2-\frac{d}{2})\Gamma(d)}
	\\
	&=
	\int_0^1 \dx u \overline u\frac{(-1)^3\imag^{4}g^2C_A2^2}{4^3 2^1\pi^2\epsilon}F_{\mu M}^{A}(zn)T^{A'}_{AE}\ft_{\mu +}^{A'}(\sigma n + u \tau n)
\end{align}
of course, there are the two integrals, over $\tau$ and $\sigma$. first is necessary for the computation, the other one belongs to the operator itself.
i already shifted the $\tau $ integral by $+\sigma$. Now i want to subtract $\sigma$ again but add $z$
\begin{align}
	\int_{-\infty}^{0} \dx \tau \int_0^1 \dx u \overline u \ft_{\mu +}^{A'}(u zn + \overline u\sigma n + u \tau n)
	&=
	is it corr like this?
\end{align}
\ifdefined\mainprogram{}
\else
\include{end}
\fi