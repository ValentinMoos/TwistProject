\ifdefined\mainprogram{}
\else
\documentclass[10pt]{article}

%packages +  bib
\usepackage[left=2.5cm,right=2.5cm,bottom=2cm,top=3.0cm]{geometry}
\usepackage{amsmath}      %std package for many operators
\usepackage{amssymb}      %symbolds i guess
\usepackage{bbold}	  %identity matrix
\usepackage{ dsfont }	  % i have no idea
\usepackage{color}              % for comments
\usepackage{xcolor}	   %for additional color, can be deleted in the end i think!!!s
\usepackage{physics}          % bra ket notation
\usepackage{slashed}          %for slash notation
\usepackage{fancyhdr}       %allows chance of layout
\usepackage[subfigure]{tocloft}    %TOC layout
\usepackage{indentfirst}      %for parindent after new section
\usepackage{graphicx,subfigure}	  %for uni logo
\usepackage{setspace}      %for spacing between the lines
\usepackage{simplewick}   %for wick contractions
\usepackage[compat=1.1.0]{tikz-feynman} %SM diagrams
\usepackage[title,titletoc]{appendix} %appendix written in table of contents
\usepackage[sorting=none]{biblatex}  %numbers of citation are how they appear in tex
\bibliography{literature.bib}

%Layout including how to enumberate equations
\setlength\parindent{0.8cm}
\setlength{\cftsecnumwidth}{35pt}
\setlength{\cftsubsecnumwidth}{35pt}
\pagestyle{fancy} %benutzerdef
\fancyhf{}
\fancyhead[R]{\thepage}
\renewcommand{\headrulewidth}{0pt}

\renewcommand{\thesection}{\Roman{section}} % i removed the ``.'' in the end ( in subsection it is still there) for references. It should work this way.
\renewcommand{\thesubsection}{\Alph{subsection}}
\renewcommand{\thesubsubsection}{\Alph{subsection}.\roman{subsubsection}}



\numberwithin{equation}{section}
\renewcommand{\theequation}{\arabic{section}.\arabic{equation}}

\renewcommand{\baselinestretch}{1.25}

%Additional Layout matters
\newcommand{\remark}[1]{\newline\newline \emph{#1} \newline\newline}

%dates


%Other / Names
\newcommand{\Mat}{\text{\emph{Mathematica}}}


%References
\newcommand{\howtowriteequation}{eq.$~$}
\newcommand{\cref}[1]{eq.$~$(#1)} % cite equation from other paper ref
\newcommand{\eref}[1]{\howtowriteequation (\ref{#1})}
\newcommand{\doubleref}[2]{\howtowriteequation (\ref{#1}, \ref{#2})}
\newcommand{\tripleref}[3]{\howtowriteequation (\ref{#1}, \ref{#2}, \ref{#3})}
\newcommand{\ereffromto}[2]{\howtowriteequation (\ref{#1})-(\ref{#2})}

\newcommand{\sref}[1]{section$~$\ref{#1}}
\newcommand{\Sref}[1]{Section$~$\ref{#1}}
\newcommand{\tref}[1]{table$~$\ref{#1}}
\newcommand{\fref}[1]{figure$~$\ref{#1}}
\newcommand{\aref}[1]{appendix$~$\ref{#1}}

\newcommand{\pref}[1]{page$~$\pageref{#1}}

\newcommand{\mc}[1]{\cite{#1}}

%Notes for correcting
\newcommand{\COM}[1]{\text{\textcolor{red}{#1}}}
\newcommand{\LBL}[0]{\COM{Hier}}
\newcommand{\spacee}{~~~~~~~}
%newline in formula with term exeeding length of a line
\newcommand{\nl}{\\&~~~}

%Functions and symbols
\newcommand{\gf}{G}
\newcommand{\hamiltonian}{H}
\newcommand{\flnso}{Op.}
\newcommand{\propagator}{\Delta}  % propagator symbol

%Fields (quark, gauge etc)
\newcommand{\Aqu}{B} %quantum gauge field
%operators
\newcommand{\operatorO}{O}
\newcommand{\twoq}[1]{\operatorO_q^{#1}}
\newcommand{\twoqbar}[1]{\overline \operatorO_{q}^{#1}}
\newcommand{\twog}[1]{\operatorO_{g}^{#1}}
\newcommand{\twogbar}[1]{\overline \operatorO_{g}^{#1}}

%Differential Operators
\newcommand{\dx}{\text{d}}
\newcommand{\abdif}[1]{\frac{\dx}{\dx #1  } }
\newcommand{\abdifof}[2]{\frac{\dx #1}{\dx #2  } }
\newcommand{\padif}[1]{\frac{\partial}{\partial #1  } }
\newcommand{\doublepadif}[2]{\frac{\partial^2}{\partial #1 \partial #2}}
\newcommand{\padifntimes}[2]{\frac{\partial^{#2}}{\partial #1^{#2}}}
\newcommand{\ft}{F}%field tensor
\newcommand{\cov} {D}
\newcommand{\covleft}{\overleftarrow{\cov}}
\newcommand{\covright}{\overrightarrow{\cov}}
\newcommand{\dxabc}{\left[\dx \alpha \dx \beta \dx \gamma\right]}
\newcommand{\dxabcprime}{\left[\dx \alpha' \dx \beta' \dx \gamma'\right]}
\newcommand{\dxfm}[2]{\frac{\dx^{#2}#1}{\left(2\pi\right)^{#2}}}
\newcommand{\dxfs}[2]{\dx^{#2}#1}
\newcommand{\ddx}{\dx ^d}


%Michelaneous including spinors
\newcommand{\metric}{\eta}
\newcommand{\idm}{{1}}
\newcommand{\abelian}[2]{\left[ #1, #2 \right]}
\newcommand{\expo}{e}
\newcommand{\skp}[2]{(#1|#2)}
\newcommand{\skpt}[2]{(#1\cdot#2)}
\newcommand{\spp}[1]{\langle #1 \rangle}
\newcommand{\aspp}[1]{\left[ #1 \right]}
\renewcommand{\trace}[0]{\text{Tr}}

%spinors
\newcommand{\psibar}{\overline\psi}
\newcommand{\psibarc}{\psibar_c}
\newcommand{\psibarq}{\psibar_q}
\newcommand{\psic}{\psi_c}
\newcommand{\psiq}{\psi_q}
\newcommand{\psip}{\psi_{+}}
\newcommand{\psim}{\psi_{-}}
\newcommand{\psit}{\overline\psi}
\newcommand{\chit}{\overline\chi}
\newcommand{\psipt}{\psit_{+}}
\newcommand{\psimt}{\psit_{-}}
\newcommand{\chip}{\chi_{+}}
\newcommand{\chim}{\chi_{-}}
\newcommand{\chipt}{\chit_{+}}
\newcommand{\chimt}{\chit_{-}}
\newcommand{\ispace}{~\hspace{-3pt}}
\newcommand{\ospex}{O}
\newcommand{\spideriv}[2]{\frac{#1\partial}{\partial #2}}
\newcommand{\spiderivb}[3]{\left(\spideriv{#1}{#2}\right)^{#3}}
\newcommand{\fbar}{\overline f}
\newcommand{\SX}{J}

%%Spinor Combinations
\newcommand{\lol}{\lambda\overline\lambda}
\newcommand{\mom}{\mu\overline\mu}
\newcommand{\lom}{\lambda\overline\mu}
\newcommand{\mol}{\mu\overline\lambda}
\newcommand{\olom}{\overline\lambda\overline\mu}
\newcommand{\omol}{\overline\mu\overline\lambda}
\newcommand{\om}{\overline\mu}
\newcommand{\ol}{\overline\lambda}
%skp that appear quite often such that the order is consistent
\newcommand{\skpyP}{\skp{y}{P}}
\newcommand{\skpPy}{\skpyP}

\newcommand{\skpyS}{\skp{y}{S}}
\newcommand{\skpSy}{\skpyS}
% q Bar, bold B , pplus, nbar
\newcommand{\qbar}{\overline q}
\newcommand{\bb}{\boldsymbol{b}}
\newcommand{\pp}{p_{+}}
\newcommand{\pphat}{\hat p_{+}}
\newcommand{\nbar}{\tilde n}
%Imaginary Unity, coupling: 
\newcommand{\imag}{\text{i}}
\newcommand{\as}{\alpha_s}
%Operators
%Wilson line, pretzelosity , Matrix element PS
\newcommand{\brez}{h_{1T}^{\bot}}
\newcommand{\wil}[2]{\left[#1,#2 \right]}
\newcommand{\myState}{P,S}
\newcommand{\fme}[1]{\bra{\myState}#1\ket{\myState}}
\newcommand{\twist}{T}
%O Gamma operators
\newcommand{\oga}{O ^{\Gamma  }}
\newcommand{\ogmu}{O^{\gamma^\mu}}
\newcommand{\ogmugfive}{O^{\gamma^\mu \gamma^5}}
\newcommand{\osigma}{O^{\imag \sigma ^{\mu \nu }\gamma_5}}
\newcommand{\ogplus}{O^{\gamma^{+}}}
\newcommand{\ogplusgfive}{O^{\gamma^{+} \gamma^5}}
\newcommand{\osigmaplus}{O^{\imag \sigma ^{\alpha + }\gamma_5}}
\newcommand{\osigmavarplus}[1]{O^{\imag \sigma ^{#1 + }\gamma_5}}
%3point operators
\newcommand{\ttp}{\mathcal{T}^\Gamma}
\newcommand{\ttpg}{\mathcal{T}^{\gamma_+}}
\newcommand{\ttpgg}{\mathcal{T}^{\gamma_+\gamma_5}}
\newcommand{\ttpsg}{\mathcal{T}^{\imag \sigma^{\alpha + }\gamma^5}}
\newcommand{\ttpsgadaptive}[3]{\mathcal{T}^{\imag \sigma^{#1 #2}\gamma^5}_{#3}}
%matrix elements of 3 point operators
\newcommand{\Dt}{\Delta T}
\newcommand{\Dtx}{\Dt (x_1,x_2,x_3)}
\newcommand{\Dtt}{\Delta \tilde T}
\newcommand{\dttg}{\delta \tilde T_g}
\newcommand{\dtte}{\delta \tilde T_\expsilon}
\newcommand{\dTe}{\delta T _\epsilon}
\newcommand{\dTex}{\delta T _\epsilon(x_1,x_2,x_3)}
\newcommand{\dTg}{\delta T _g}
\newcommand{\dTgx}{\delta T _g(x_1,x_2,x_3)}
\newcommand{\xonetwothree}{x_{1,2,3}}
%U gamma operators or full TMD s
\newcommand{\uga}{\mathcal{U}^{\Gamma}}
\newcommand{\udis}{\mathcal{U}_{DIS}^{\Gamma}}
\newcommand{\udy}{\mathcal{U}_{DY}^{\Gamma}}
\newcommand{\udygplus}{\mathcal{U}_{DY}^{\gamma^+}}
\newcommand{\udygplusgfive}{\mathcal{U}_{DY}^{\gamma^+\gamma_5}}
\newcommand{\udysg}{\mathcal{U}_{DY}^{\imag \sigma^{\alpha +}_{T} \gamma_5}}

\newcommand{\nameofO}{O}	
\newcommand{\ogt}{\nameofO_{TMD}^{\Gamma}}
\newcommand{\ogtfields}[2]{\nameofO_{#1}^{#2}}
\newcommand{\ogpt}{\nameofO_{TMD}^{\gamma_+}}

%phi distributions
\newcommand{\phiqh}{\Phi_{q\leftarrow h}}
\newcommand{\phiqhij}{\Phi_{q\leftarrow h,ij}}
\newcommand{\phiG}{\Phi_{q\leftarrow h}^{[\Gamma]}}
\newcommand{\phig}{\Phi_{q\leftarrow h}^{[\gamma^+]}}
\newcommand{\phigg}{\Phi_{q\leftarrow h}^{[\gamma^+ \gamma_5]}}
\newcommand{\phisg}[1]{\Phi_{q\leftarrow h}^{[\sigma^{#1 +}\gamma_5]}}

%Parametrization fucntions
\newcommand{\paraA}{A}
\newcommand{\paraB}{B}


%Graphics

\usepackage[]{hyperref}

\begin{document}

\fi

\section{Comments}
Note that I use "good components", that is at some point in the computation I multiply (from the middle-direction of the operator) by $\gamma^+$, and by anticommuting find usually a term that is proportional to $\gamma^+$ and a term that drops. To my understanding i can do this since the full operator will contain a Gamma struncture inside that will contain this $\gamma^+$. This should only affect loop diagrams containing gamma structure (quark propagators).

\section{gluon diagrams}

Here i present the Diagrams for twist 2 with the intermediate computation performed in the Mathematica files.
I will write the expressions for the diagrams, then isolte the results like it is done in Mathematica.
The variables are
$z1$=zn , $z2$=$\sigma n $, $z3=\tau n$ (integrate over)
Note that the propagator symbol here usually is only the "one over distance squared" term, like in Mathematica, no factor of pi, 4 or Gamma included.
\subsubsection{Diagram A}
\begin{align}
&\int_{z_2}^{z_1}\dx z_3\ft_{\mu +}^{A}(z_1)\imag g B_+^B(z3)  T^B_{AC}\ft _{\nu N}^D T^D_{CE}\bigg(-\imag g \int \ddx xA_\iota^{A'}(x) \partial_{x^\alpha} B_\beta^{B'} (x) B_\gamma^{C'}(x) \bigg) v^{\iota \alpha\beta\gamma}_{A'B'C'}
\end{align}

The common "prefactor" is
\begin{align}
	\int_{z2}^{z1} \dx z3 \int \ddx x A_\iota^{A'}(x) (- \imag ^2) \frac{\Gamma^2\left(\frac{d}{2}-1\right)}{(4\pi^2)^2}
\end{align}
The two contrations are
\begin{align}
	&\partial_{z1^{\mu}}\partial_{x^{\alpha}}\propagator_{M\beta}(z_1 -x )\propagator_{+\gamma}(z_3 -x )
	\nl
	+\partial_{z1^{\mu}}\propagator_{M\gamma}(z_1 -x )\partial_{x^{\alpha}}\propagator_{+\beta}(z_3 -x )
\end{align}

The color factor is: all structure function structure, $f$ and $T$ with the $\delta$-s coming from propagators computed
The color factor is( the notation +1-1 indicates the first contraction is + the second is -, it does NOT mean the color factor is zero!):
\begin{align}
	i T^{D}_{A'E}C_A(+1-1)
\end{align}
Plug this into mathematica.
Output: (be aware that now $z_i$ is only $z$, $\sigma$ and $\tau$ without $n$ )
prefactor
\begin{align}
	\frac{g^2}{8 \pi^2 \epsilon} \ft_{\nu +}^D T^D_{A'E}
\end{align}
integral term
\begin{align}
	&\int_0^1 \dx u \int_{z_3 -z_2}^{0} \dx z_3 u \partial_+ \ft_{\mu +}(z_1 n + \overline u z_3 n)
	\\
	&=\int_0^1 \dx u \frac{u}{\overline u} \left(\ft_{\mu +}^{A'}(z_1 n) -\ft_{\mu +}^{A'}(z_2n + u(z_1-z_2) n) \right)
\end{align}

\subsubsection{Diagram A'}
\begin{align}
&\int_{-\infty}^{z_2}\dx z_3\ft_{\mu +}^{A}(z_1)\ft _{\nu N}^D(z_2) T^D_{AC}\imag g B_+^B(z3)  T^B_{CE}\bigg(-\imag g \int \ddx xA_\iota^{A'}(x) \partial_{x^\alpha} B_\beta^{B'} (x) B_\gamma^{C'}(x) \bigg) v^{\iota \alpha\beta\gamma}_{A'B'C'}
\end{align}

The common "prefactor" is
\begin{align}
\int_{z2}^{z1} \dx z3 \int \ddx x A_\iota^{A'}(x) (- \imag ^2) \frac{\Gamma^2\left(\frac{d}{2}-1\right)}{(4\pi^2)^2}
\end{align}
The two contrations are
\begin{align}
&\partial_{z1^{\mu}}\partial_{x^{\alpha}}\propagator_{M\beta}(z_1 -x )\propagator_{+\gamma}(z_3 -x )
\nl
+\partial_{z1^{\mu}}\propagator_{M\gamma}(z_1 -x )\partial_{x^{\alpha}}\propagator_{+\beta}(z_3 -x )
\end{align}

The color factor is:
\begin{align}
i T^{D}_{A'E}\frac{C_A}{2}(+1-1)
\end{align}
Plug this into mathematica.
Output: (be aware that now $z_i$ is only $z$, $\sigma$ and $\tau$ without $n$ )
prefactor
\begin{align}
\frac{g^2}{16 \pi^2 \epsilon} \ft_{\nu +}^D T^D_{A'E}
\end{align}
integral term
\begin{align}
&\int_0^1 \dx u \int_{-\infty}^{z_3 -z_2} \dx z_3 u \partial_+ \ft_{\mu +}(z_1 n + \overline u z_3 n)
\\
&=\left(1- \ln\left(\frac{\imag \pphat}{\delta}\right)\right)\ft_{\mu +}^{A'}(z_1n)+\int_0^1 \dx u \frac{u}{\overline u} \left(\ft_{\mu +}^{A'}(z_1 n+\overline u (z_2-z_1)n) -\ft_{\mu +}^{A'}(z_1 n) \right)
\end{align}

\subsubsection{Diagram B}
\begin{align}
&\int_{z_2}^{z_1}\dx z_3\ft_{\mu +}^{A}(z_1)\imag g B_+^B(z3)  T^B_{AC}\ft _{\nu N}^D(z_2) T^D_{CE}\bigg(-\imag g \int \ddx xA_\iota^{A'}(x) \partial_{x^\alpha} B_\beta^{B'} (x) B_\gamma^{C'}(x) \bigg) v^{\iota \alpha\beta\gamma}_{A'B'C'}
\end{align}

The common "prefactor" is
\begin{align}
\int_{z2}^{z1} \dx z3 \int \ddx x A_\iota^{A'}(x) (- \imag ^2) \frac{\Gamma^2\left(\frac{d}{2}-1\right)}{(4\pi^2)^2}
\end{align}
The two contrations are
\begin{align}
&\partial_{z2^{\mu}}\partial_{x^{\alpha}}\propagator_{M\beta}(z_2 -x )\propagator_{+\gamma}(z_3 -x )
\nl
+\partial_{z2^{\mu}}\propagator_{M\gamma}(z_2 -x )\partial_{x^{\alpha}}\propagator_{+\beta}(z_3 -x )
\end{align}

The color factor is:
\begin{align}
i T^{A'}_{AE}\frac{C_A}{2}(+1-1)
\end{align}
Plug this into mathematica.
Output: (be aware that now $z_i$ is only $z$, $\sigma$ and $\tau$ without $n$ )
prefactor
\begin{align}
\frac{g^2}{16 \pi^2 \epsilon} \ft_{\mu +}^A T^{A'}_{AE}
\end{align}
integral term
\begin{align}
&\int_0^1 \dx u \int_{z_1 -z_2}^{0} \dx z_3 u \partial_+ \ft_{\nu +}(z_2 n + \overline u z_3 n)
\\
&=\int_0^1 \dx u \frac{u}{\overline u} \left(\ft_{\nu +}^{A'}(z_2n+\overline u(z_1-z_2)n)-\ft_{\nu +}^{A'}(z_2n)\right)
\end{align}

\subsubsection{Diagram B'}
\begin{align}
&\int_{-\infty}^{z_2}\dx z_3\ft_{\mu +}^{A}(z_1)\ft _{\nu N}^D(z_2) T^D_{AC}\imag g B_+^B(z3)  T^B_{CE}\bigg(-\imag g \int \ddx xA_\iota^{A'}(x) \partial_{x^\alpha} B_\beta^{B'} (x) B_\gamma^{C'}(x) \bigg) v^{\iota \alpha\beta\gamma}_{A'B'C'}
\end{align}

The common "prefactor" is
\begin{align}
\int_{z2}^{z1} \dx z3 \int \ddx x A_\iota^{A'}(x) (- \imag ^2) \frac{\Gamma^2\left(\frac{d}{2}-1\right)}{(4\pi^2)^2}
\end{align}
The two contrations are
\begin{align}
&\partial_{z_2^{\mu}}\partial_{x^{\alpha}}\propagator_{M\beta}(z_2 -x )\propagator_{+\gamma}(z_3 -x )
\nl
+\partial_{z_A2^{\mu}}\propagator_{M\gamma}(z_2 -x )\partial_{x^{\alpha}}\propagator_{+\beta}(z_3 -x )
\end{align}

The color factor is:
\begin{align}
i T^{A'}_{AE}\frac{C_A}{2}(+1-1)
\end{align}
Plug this into mathematica.
Output: (be aware that now $z_i$ is only $z$, $\sigma$ and $\tau$ without $n$ )
prefactor
\begin{align}
\frac{g^2}{16 \pi^2 \epsilon} \ft_{\mu +}^A T^{A'}_{AE}
\end{align}
integral term
\begin{align}
&\int_0^1 \dx u \int_{-\infty}^{0} \dx z_3 u \partial_+ \ft_{\nu +}(z_2 n + \overline u z_3 n)
\\
&=\left(1- \ln\left(\frac{\imag \pphat}{\delta}\right)\right)\ft_{\nu +}^{A'}(z_2n)
\end{align}

\section{quark Twist 2}

\subsection{Diagram D}
\begin{align}
\int_{z_2}^{z_1} \dx z_3 \qbar^A(z_1) \imag g B_+^Z T^Z_{AC} \ft_{\mu +}^D(z_2) T^D_{CE}(-\imag g \int \ddx x A^{A'}_\iota \partial_\alpha B_\beta^{B'}D_\gamma^{C'})v^{\iota\alpha\beta\gamma}_{A'B'C'}
\end{align}
common prefactor:
\begin{align}
\int_{z_2}^{z_1} \dx z_3  \int \ddx x \qbar(z_1) A^{A'}_\iota(x)(-\imag^2)\frac{\Gamma^2(\frac{d}{2}-1)}{\left(4\pi^2\right)^{2}}
\end{align}
Two contractions:
\begin{align}
	&\partial_{x^\alpha} \partial_{z_1^\mu} \propagator_{M\beta}(z_2 -x) \propagator_{\gamma +}(z_3-x)
	\nl
	+\partial_{z_1^\mu} \propagator_{M\gamma}(z_2 -x) \partial_{x^\alpha}\propagator_{\beta +}(z_3-x)
\end{align}
Color factor
\begin{align}
	-\imag \frac{C_F}{2} T^{A'}_{AD}
\end{align}
Output:
prefactor:
\begin{align}
	-\frac{C_F g^2}{16 \pi^2 \epsilon}\qbar(z_1n)^A T^{A'}_{AD}
\end{align}
integral term
\begin{align}
	&\int_0^1 \dx u \int_0^{z_1-z_2} \dx z_3 u \partial_+ \ft_{\mu +}^{A'} 
	\\
	&=\int _0^1 \dx u \frac{u}{\overline u} \left(\ft_{\mu +}^{A'}(z_1n+u(z_2-z_1)n)-\ft_{\mu +}^{A'}(z_2n)\right) 
\end{align}



\subsection{Diagram H}
\begin{align}
\int_{z_2}^{z_1} \dx z_3 \qbar^A(z_1) \ft_{\mu +}^D(z_2) T^D_{AC}\imag g B_+^Z T^Z_{CE} (-\imag g \int \ddx x A^{A'}_\iota \partial_\alpha B_\beta^{B'}D_\gamma^{C'})v^{\iota\alpha\beta\gamma}_{A'B'C'}
\end{align}
common prefactor:
\begin{align}
\int_{-\infty}^{z_2} \dx z_3  \int \ddx x \qbar(z_1) A^{A'}_\iota(x)(-\imag^2)\frac{\Gamma^2(\frac{d}{2}-1)}{\left(4\pi^2\right)^{2}}
\end{align}
Two contractions:
\begin{align}
&\partial_{x^\alpha} \partial_{z_1^\mu} \propagator_{M\beta}(z_2 -x) \propagator_{\gamma +}(z_3-x)
\nl
+\partial_{z_1^\mu} \propagator_{M\gamma}(z_2 -x) \partial_{x^\alpha}\propagator_{\beta +}(z_3-x)
\end{align}
Color factor
\begin{align}
+\imag \frac{C_F}{2} T^{A'}_{AD}
\end{align}
Output:
prefactor:
\begin{align}
\frac{C_F g^2}{16 \pi^2 \epsilon}\qbar(z_1n)^A T^{A'}_{AD}
\end{align}
integral term
\begin{align}
&\int_0^1 \dx u \int_{-\infty}^{0} \dx z_3 u \partial_+ \ft_{\mu +}^{A'} 
\\
&=\left(1-\ln\left(\frac{\imag \pphat(z_2)}{\delta}\right)\right)\ft_{\mu +}^{A'}(z_2) 
\end{align}



\subsection{Diagram E}
Note that in the quark propagator there is hidden a factor if $\imag$!
\begin{align}
\int_{z_2}^{z_1} \dx z_3 \qbar^A(z_1) \imag g B_+^B T^B_{AC} \ft_{\mu +}^D(z_2) T^D_{CE} (+\imag g \int \ddx x \qbar^X \gamma^\nu B_\nu^Z T^Z_{XY} q^Y)
\end{align}
common prefactor:
\begin{align}
\int_{z_2}^{z_1} \dx z_3  \int \ddx x \imag ^3 g^2  \frac{\Gamma^(\frac{d}{2})\Gamma^(\frac{d}{2}-1)}{8\pi^4}\ft_{\mu +}^D
\end{align}
One contraction:
\begin{align}
&\qprop(z_1 -x) \propagator_{\nu +}(z_3-x)
\end{align}
Color factor
\begin{align}
\frac{C_F} T^{D}_{XE}
\end{align}
Output:
prefactor:
\begin{align}
-\frac{ C_F g^2}{8 \pi^2 \epsilon}\ft_{\mu +}^D(z_2n) T^{D}_{ZE}
\end{align}
integral term
\begin{align}
&\int_0^1 \dx u \int_{0}^{z_1-z_2} \dx z_3 u \partial_+ \qbar^Z(z_1+\overline u z_3) 
\\
&=\int_0^1 \dx u \frac{u}{\overline u}\left(\qbar^Z((z_2+u(z_1-z_2))n)-\qbar^Z(z_1n)\right) 
\end{align}


\subsection{Diagram G}
Note that in the quark propagator there is hidden a factor if $\imag$!
\begin{align}
\int_{-\infty}^{z_2} \dx z_3 \qbar^A(z_1) \ft_{\mu +}^D(z_2) T^D_{AC} \imag g B_+^B T^B_{CE}  (+\imag g \int \ddx x \qbar^X \gamma^\nu B_\nu^Z T^Z_{XY} q^Y)
\end{align}
common prefactor:
\begin{align}
\int_{z_2}^{z_1} \dx z_3  \int \ddx x \imag ^3 g^2  \frac{\Gamma^(\frac{d}{2})\Gamma^(\frac{d}{2}-1)}{8\pi^4}\ft_{\mu +}^D
\end{align}
One contraction:
\begin{align}
&\qprop(z_1 -x) \propagator_{\nu +}(z_3-x)
\end{align}
Color factor
\begin{align}
\frac{C_F}{2} T^{D}_{XE}
\end{align}
Output:
prefactor:
\begin{align}
-\frac{ C_F g^2}{16 \pi^2 \epsilon}\ft_{\mu +}^D(z_2n) T^{D}_{ZE}
\end{align}
integral term
\begin{align}
&\int_0^1 \dx u \int_{-\infty}^{z_2-z_1} \dx z_3 u \partial_+ \qbar^Z(z_1+\overline u z_3) 
\\
&=\int_0^1 \dx \frac{u}{\overline u} \left(\qbar^Z((z_2+u(z_1-z_2))n)-\qbar^Z(z_1n)\right)
\nl
+\left(1-\ln\left(\frac{\imag \pphat(z_1)}{\delta}\right)\right)\qbar^Z(z_1n)
\end{align}

\subsection{Diagram F}
\begin{align}
\qbar^A(z_1) f^{DXY} B_\mu^X(z_2)B_+^Y(z_2) T^D_{AC}  (-\imag g \int \ddx x A^{A'}_\iota \partial_\alpha B_\beta^{B'}D_\gamma^{C'})v^{\iota\alpha\beta\gamma}_{A'B'C'}
\end{align}
common prefactor:
\begin{align}
\int \ddx x -\imag g^2  \frac{\Gamma^2(\frac{d}{2}-1)}{16\pi^4}\qbar^A(z_1)
\end{align}
contractions:
\begin{align}
&\partial_{x^\alpha}\propagator_{\mu \beta}(z_2-x)\propagator_{\gamma +}(z_2-x)
\nl 
+\propagator_{\mu \gamma}(z_2-x)\partial_{x^\alpha}\propagator_{\beta +}(z_2-x)
\end{align}
Color factor
\begin{align}
{C_F} T^{A'}_{AC}
\end{align}
Output:
prefactor:
\begin{align}
-\frac{ C_F g^2}{4 \pi^2 \epsilon}\qbar^A(z_1) T^{A'}_{AE}
\end{align}
integral term( here the integral over u is trivial and gives factor $\frac12$)
\begin{align}
\ft_{\mu +}^{A'}(z_2n)
\end{align}

\subsection{Diagram C}
\begin{align}
\qbar^A(z_1) g f^{DLK} B_\mu^L(z_2)B_+^K(z_2) T^D_{AE}   (+\imag g \int \ddx x \qbar^X \gamma^\nu B_\nu^Z T^Z_{XY} q^Y)
\end{align}
common prefactor:
\begin{align}
\int \ddx x \imag g^2  \frac{\Gamma(\frac{d}{2}-1)\Gamma(\frac{d}{2})}{8\pi^4}\qbar^X(x)
\end{align}
contractions:( i could drop the second one from the start)
\begin{align}
&\qprop(z_1-x)\propagator_{\nu +}(z_2-x)A_\mu^L(z_2)\gamma^\nu
\nl 
+\qprop(z_1-x)\propagator_{\nu \mu}(z_2-x)A_+^L(z_2)\gamma^\nu
\end{align}
Color factor
\begin{align}
\frac{C_F}{2} T^{L}_{XE}(+1-1)
\end{align}
Output:
prefactor:
\begin{align}
-\frac{ C_F g^2}{16 \pi^2 \epsilon}
\end{align}
integral term
\begin{align}
\int_0^1 \dx u& u \qbar^X(uz_1n+\overline u z_2n) \ppl A_\mu ^L(z_2 n)T^L_{XE}
\\
&=\int _0^1 \dx u (z_1-z_2)^{-1}\padif{u}\qbar(z_{2\rightarrow1}^un)A_\mu(z_2n)
\\
&=
(z_1-z_2)^{-1}\left(-\int _0^1 \dx \qbar(z_{2\rightarrow1}^un) +\qbar(z_1n)\right)A_\mu(z_2n)
\\
&=
(z_1-z_2)^{-1}\int _{-\infty}^{z_2} \dx \beta \left(\int _0^1 \dx \qbar(z_{2\rightarrow1}^un) -\qbar(z_1n)\right)\ft_{\mu+}(\beta n)
\end{align}


\subsection{Diagram K}
\begin{align}
\qbar^A(z_1) \ft_{\mu +}^D(z_2) T^D_{AE}   (+\imag g \int \ddx x \qbar^X \gamma^\nu B_\nu^Z T^Z_{XY} q^Y)
\end{align}
common prefactor:
\begin{align}
\int \ddx x \imag g \frac{\Gamma(\frac{d}{2}-1)\Gamma(\frac{d}{2})}{8\pi^4}\qbar^X(x)
\end{align}
\begin{align}
\qprop(z_1-x)\partial_{z_1^\mu}\propagator_{\nu M}(z_2-x)\gamma^\nu
\end{align}
Color factor
\begin{align}
	C_F T^{L}_{XC}
\end{align}
Output:
prefactor:
\begin{align}
\frac{- \imag  C_F g}{32 \pi^2 \epsilon}
\end{align}
integral term
\begin{align}
\int_0^1 \dx u u \overline u \qbar^C(uz_1n+\overline u z_2n)\slashed \partial \partial_+ \gamma^\mu 
\end{align}
Further evolution:
\begin{align}
\int_0^1 \dx u u \overline u \qbar^C(uz_1n+\overline u z_2n)\slashed \partial \partial_+ \gamma^\mu
\\
&=
- \int_0^1 \dx u (1-2u) \qbar^C(uz_1n+\overline u z_2n)\slashed \partial  \gamma^\mu
\\
&=
\imag g  \int_0^1 \dx u (1-2u) \qbar^C(uz_1n+\overline u z_2n)\slashed A(uz_1n+\overline u z_2n)  \gamma^\mu
\\
&=
\imag g  \int_0^1 \dx u (1-2u) \qbar^C(uz_1n+\overline u z_2n)\int_{-\infty}^{uz_1+\overline u z_2} \dx \beta \ft_{+\nu}(\beta n) \gamma^\nu \gamma^\mu
\\
&=
- \imag g  \int_0^1 \dx u (1-2u) \qbar^C(uz_1n+\overline u z_2n)\left( \int_{-\infty}^{z_2} + \int_{z_2}^{uz_1+\overline u z_2} \right) \dx \beta \ft_{\nu +}(\beta n) \gamma^\nu \gamma^\mu
\end{align}

\subsection{Diagram B}
\begin{align}
\qbar^A(z_1) \ft_{\mu +}^D(z_2) T^D_{AC}   (+\imag g \int \ddx x \qbar^X \gamma^\nu B_\nu^Z T^Z_{XY} q^Y)(+\imag g \int \ddx y \qbar^O \gamma^\nu B_\nu^P T^P_{OV} q^V)
\end{align}
common prefactor:
\begin{align}
\int \ddx x \int \ddx y \imag^2 g^2 \frac{\Gamma(\frac{d}{2}-1)\Gamma^2(\frac{d}{2})}{16\pi^6}
\end{align}
\begin{align}
(1)=&\qprop(z_1-x)\partial_{z_1^\mu}\propagator_{\eta M}(z_2-y)\qprop(x-y)\qbar^O(y)\gamma^\eta \gamma^\nu
\\
(2)=&\qprop(z_1-y)\partial_{z_1^\mu}\propagator_{\nu M}(z_2-x)\qprop(y-x)\qbar^X(x)\gamma^\nu\gamma^\eta 
\end{align}
Color factor
\begin{align}
(1)&=\frac{C_F}{2} T^{Z}_{OC}
\\
(2)&=\frac{C_F}{2} T^{P}_{XC}
\end{align}
Output:
prefactor(blue factor 2 is symmetry factor):
\begin{align}
{\color{blue}2}\frac{ - C_F g^2 \Gamma(-\epsilon)}{32 \pi^2} &=\frac{ C_F \as}{\epsilon}
\end{align}
integral term
\begin{align}
\int \dxabc &\frac{\alpha \beta \gamma}{\left(\alpha \beta + \beta \gamma + \gamma \alpha\right)^5} \qbar^O(\frac{z_1 \gamma\alpha +z_2 \beta (\gamma + \alpha)}{\alpha \beta + \beta \gamma + \gamma \alpha}n)T^{Z}_{OC}
\nl
\bigg\{ (z_1-z_2)\alpha \beta\gamma (\beta + \gamma) \gamma^\mu \gamma^\nu \ppr^2 
+ (z_1-z_2)\alpha\beta\gamma (\alpha + \gamma) \gamma^\mu \gamma^\nu \ppl^2 
\nl
+(\beta \gamma + \alpha (\beta +\gamma) )((\alpha(\beta-2\gamma)+\beta\gamma)\gamma^\mu \gamma^\nu -\alpha \gamma \gamma^\nu \gamma^\mu)\ppl 
\nl
+ (z_1-z_2)\alpha\beta(\alpha(\beta+\gamma)+\gamma(\beta+2\gamma))\gamma^\mu \gamma^\nu \ppl \ppr
\nl
-(\beta\gamma + \alpha (\beta +\gamma))( (-\beta\gamma+2\alpha (\beta+\gamma))\gamma^\mu \gamma^\nu  - \beta \gamma \gamma^\nu \gamma^\mu )\ppr
	 \bigg\}A^Z_\nu (\frac{z_1(\gamma +\beta)\alpha+z_2\gamma \beta}{\alpha \beta + \beta \gamma + \gamma \alpha}) 
\\
&=
\int \dxabc \frac{\alpha \beta \gamma \Lambda^2}{\Lambda^5} \qbar^O(z_{21}^{\beta ' }n)T^{Z}_{OC}
\nl
\bigg\{ (z_1-z_2)\ap \apbar \gamma^\mu \gamma^\nu \ppr^2 
+ (z_1-z_2)\beta' \overline{\beta '} \gamma^\mu \gamma^\nu \ppl^2 
\nl
+(1-3\beta')\gamma^\mu \gamma^\nu -\beta' \gamma^\nu \gamma^\mu \ppl 
\nl
+ (z_1-z_2)(\gamma' +2\beta'\alpha')\gamma^\mu \gamma^\nu \ppl \ppr
\nl
+(3\alpha'-2)\gamma^\mu \gamma^\nu  + \alpha' \gamma^\nu \gamma^\mu \ppr
\bigg\}A^Z_\nu (z_{12}^{\alpha'}) 
\end{align}
with
\begin{align}
	z_{ij}^u
	&=
	z_i+u(z_j-z_i)
\end{align}
"from $i$ to $j$ along variable $u$"

change to primed variables:
\begin{align}
	(\alpha',\beta',\gamma') \rightarrow (\beta,\alpha,\gamma) 
\end{align}
prefactor
\begin{align}
\frac{ C_F \as}{\epsilon}
\end{align}

integral term
\begin{align}
\int \dxabc & \qbar^O(z_{21}^{\alpha  }n)T^{Z}_{OC}
\bigg\{ (z_1-z_2)\beta \overline{\beta} \gamma^\mu \gamma^\nu \ppr^2 
+ (z_1-z_2)\alpha \overline{\alpha} \gamma^\mu \gamma^\nu \ppl^2 
\nl
+\left((1-3\alpha)\gamma^\mu \gamma^\nu -\alpha \gamma^\nu \gamma^\mu \right)\ppl 
+ (z_1-z_2)(\gamma +2\alpha\beta)\gamma^\mu \gamma^\nu \ppl \ppr
\nl
+\left((3\beta-2)\gamma^\mu \gamma^\nu  + \beta \gamma^\nu \gamma^\mu \right)\ppr
\bigg\}A^Z_\nu (z_{12}^{\beta}) 
\end{align}
useful:
\begin{align}
	\partial_+ f(z_{ij}^\lambda n) &= -(z_i-z_j)^{-1}\partial_\lambda f(z_{ij}^\lambda n)
	\\
	A_\alpha(z_{ij}^\lambda n) &= (z_i-z_j)\int_\Pi^\lambda \dx \sigma \ft_{\alpha +}(z_{ij}^\sigma n)
	\\
	x
\end{align}
$\Pi$ would be a $\Pi$oint of reference, usually $\pm \infty$
refine the result term by term:
\begin{align}
	\int \dxabc (z_1-z_2)\beta \overline{\beta} \ppr^2 A^Z_\nu (z_{12}^{\beta}) 
	&=
	\int \dxabc (1-2\beta) \ft^Z_{\nu+ } (z_{12}^{\beta})
	\\
	\int\dxabc \qbar(z_{21}^\alpha n) \ppl^2 \alpha \overline{\alpha}(z_1-z_2)A_\nu(z_{12}^\beta)
	&=
	\int\dxabc  \left( 2\qbar(z_{21}^\alpha n) - \qbar(z_1n) -\qbar(z_2n) \right) \int_\Pi^\beta  \ft_{\nu+} (z_{12}^\sigma n)
	\\
	\int\dxabc \qbar(z_{21}^\alpha n) \ppl (-\alpha) A_\nu(z_{12}^\beta)
	&=
	\int\dxabc  \left(\qbar(z_1n) - \qbar(z_{21}^\alpha n)  \right) \int_\Pi^\beta  \ft_{\nu+} (z_{12}^\sigma n)
	\\
	\int\dxabc \qbar(z_{21}^\alpha n) \ppl (1-3\alpha) A_\nu(z_{12}^\beta)
	&=
	\int\dxabc  \left(\qbar(z_2n)+2\qbar(z_1n) - 3 \qbar(z_{21}^\alpha n)  \right) \int_\Pi^\beta  \ft_{\nu+} (z_{12}^\sigma n)
	\\
	\int\dxabc \qbar(z_{21}^\alpha n) \ppl (z_1-z_2) (1-\alpha-\beta +2 \alpha\beta)
	&=
	\int\dxabc  \left(-\qbar(z_2n)\overline{\beta}+ \qbar(z_1n) \beta  - \qbar(z_{21}^\alpha n)(2\beta-1)  \right)
\end{align}
the other terms are already in good form by derivative acting on gluon field.
resumming
\begin{align}
	B&= \int \dxabc \bigg\{ \gamma_\mu \gamma^\nu \big[
	(1-2\beta)
	\big] \bigg\}
\end{align}
\subsection{Diagram A}
\begin{align}
\qbar^A(z_1) \ft_{\mu +}^D(z_2) T^D_{AC}   (+\imag g \int \ddx x \qbar^X \gamma^\nu B_\nu^Z T^Z_{XY} q^Y)(-\imag g \int \ddx y A^{A'}_\iota \partial_\alpha B_\beta^{B'}D_\gamma^{C'})v^{\iota\alpha\beta\gamma}_{A'B'C'}
\end{align}
common prefactor:
\begin{align}
\int \ddx x \int \ddx y -\imag^2 g^2 \frac{\Gamma^2(\frac{d}{2}-1)\Gamma(\frac{d}{2})}{32\pi^6} \qbar^X(x)\gamma^\theta A_\iota^{A'}(y)
\end{align}
\begin{align}
&\qprop(z_1-x)\partial_{y^\alpha}\partial_{z_2^\mu}\propagator_{\beta M}(z_2-y)\propagator_{\theta \gamma}(x-y)
\nl
+\qprop(z_1-x)\partial_{z_2^\mu}\propagator_{\gamma M}(z_2-y)\partial_{y^\alpha}\propagator_{\theta \beta}(x-y) 
\end{align}
Color factor
\begin{align}
-\imag\frac{C_F}{2} T^{A'}_{XZ}(1-1)
\end{align}
Output(main term from main computation, there is an additional term right below):
prefactor:
\begin{align}
\frac{g^2 C_F\Gamma(-\epsilon)}{32\pi^2}&=-\frac{C_F\as}{2\epsilon} 
\end{align}
integral term
\begin{align}
\int \dxabc& \frac{\alpha \beta \gamma}{\Lambda^3}\qbar^X(z_{12}^{\alpha'})\bigg\{ \bigg[ \big(-\frac{\beta(\alpha+\gamma)}{\Lambda } \ppl - \frac{\beta\gamma}{\Lambda} \ppr \big) \gamma_\mu \gamma^\nu +  \ppl \gamma^\nu \gamma_\mu\bigg] 
\nl
+ \bigg[  \frac{3\beta\gamma +\alpha(3\beta +\gamma)}{\Lambda} \ppl + \frac{(\beta+\gamma)(3\beta\gamma +\alpha(3\beta+\gamma))}{\gamma \Lambda} \ppr \bigg] \metric _{\mu}^\nu\bigg\} A_\nu^{A'}(z_{21}^{\beta'})
\\
&=
\int \dxabc \frac{\alpha \beta \gamma}{\Lambda^3}\qbar^X(z_{12}^{\alpha'})\bigg\{ \bigg[ \big(-\overline{\beta'} \ppl - \alpha' \ppr \big) \gamma_\mu \gamma^\nu +  \ppl \gamma^\nu \gamma_\mu\bigg] 
\nl
+ \bigg[  (3\overline{\beta'} + \beta') \ppl + \frac{(\beta+\gamma)(3\beta\gamma +\alpha(3\beta+\gamma))}{\gamma \Lambda} \ppr \bigg] \metric _{\mu}^\nu\bigg\} A_\nu^{A'}(z_{21}^{\beta'})
\\
&=
\int \dxabc \qbar^X(z_{12}^{\beta})\bigg\{ \bigg[ \big(-\overline{\alpha} \ppl - \beta \ppr \big) \gamma_\mu \gamma^\nu +  \ppl \gamma^\nu \gamma_\mu\bigg] 
\nl
+ \bigg[  (3\overline{\alpha} + \alpha) \ppl + X_{undefined} \ppr \bigg] \metric _{\mu}^\nu\bigg\} A_\nu^{A'}(z_{21}^{\alpha})
\end{align}
Then here is a bonus term, which can not be neglected. It is $\propto \metric^{\mu \alpha}$ (inside the computations first steps). The contribution is
prefactor:
\begin{align}
	\frac{g^2C_F\Gamma(-\epsilon)}{32 \pi^2}T^{A'}_{XC} &= 
	\frac{-\as C_F}{2\epsilon}T^{A'}_{XC}
\end{align}	
integral term
\begin{align}
	\int \dxabc \qbar^X(z_{12}^{\alpha'})\frac{-3\alpha \beta}{(\alpha+\gamma)\Lambda^2}\ppr  A_\mu^{A'}(z_{21}^{\beta'})
\end{align}

we can use the Anticommutator to rewrite the metric tensor:
\begin{align}
	\metric^{\vartheta\eta} &= \frac{1}{2}\left(\gamma^\vartheta \gamma^\eta + \gamma^\eta \gamma^\vartheta \right)
\end{align}
Thus in Total we have
\begin{align}
	A &=
	\int\dxabc \qbar(z_{12}^\beta)\bigg\{ 
	\gamma_\mu \gamma^\nu \big[
	+\left(\frac{1}{2}\overline{\alpha}+\alpha\right)\ppl
	+\left(-\beta+XY\right)\ppr
	\big]
	\nl
	+\gamma^\nu \gamma_\mu \big[
	\left( 1+\frac{3}{2}\overline{\alpha}+\frac{1}{2}\alpha \right)\ppl
	+XY\ppr
	\big]
	\bigg\}A_\nu^{A'}(z_{21}^{\alpha})
\end{align}
\section{Identities}
For color factor computations:
\begin{align}
	f^{ade}f^{bcd}+f^{bde}f^{cad}+f^{cde}f^{abd} &= 0
	\\
	f^{heb}(f^{ade}f^{bcd}+f^{bde}f^{cad}+f^{cde}f^{abd}) &= 0
	\\
	f^{ade}f^{heb}f^{bcd}+f^{bde}f^{heb}f^{cad}+f^{cde}f^{heb}f^{abd} &= 0
	\\
	f^{ade}f^{heb}f^{bcd}+f^{bde}f^{heb}f^{cad}+f^{cde}f^{heb}f^{abd} &= 0
	\\
	-f^{ade}f^{heb}f^{cbd}+f^{deb}f^{heb}f^{cad}-f^{adb}f^{hbe}f^{ced} &= 0
	\\
	-2f^{ade}f^{heb}f^{cbd}+C_G\delta_{dh}f^{cad} &= 0
	\\
	f^{ade}f^{heb}f^{cbd} &= \frac{C_G}{2}f^{ahc}
\end{align}

for 3 integration variables: There exists the following transformation
\begin{align}
	(\alpha, \beta, \gamma) \rightarrow (\alpha',\beta',\gamma') = (\frac{\beta\gamma}{\Lambda},\frac{\gamma\alpha}{\Lambda},\frac{\alpha\beta}{\Lambda}), ~~\Lambda = \alpha \beta + \beta \gamma + \gamma \alpha
\end{align}
with the additional constraint 
\begin{align}
	\alpha + \beta +\gamma =1
\end{align}
it is indeed a two dimensional transformation.
These variables arise naturally in the computation.
To change the variables, the Jacobi Determinant is needed to use the Transformation theorem. It should be noted that also the transformed variables naturally fulfill the condition
\begin{align}
	\alpha' + \beta ' + \gamma ' = \frac{\beta \gamma + \gamma \alpha + \alpha \beta}{\Lambda} = 1
\end{align}

and preserve the domain for each variable:
\begin{align}
	\alpha, \beta, \gamma, \in [0,1] \rightarrow \alpha', \beta', \gamma', \in [0,1]
\end{align}
The Jacobian of this (two dimensional) transformation is

\begin{align}
	\abs{\det D\Phi} &= \frac{\alpha \beta \gamma }{\Lambda ^3} 
\end{align}
which is checked in Mathematica. I also checked this by hand and found the same.
\ifdefined\mainprogram{}
\else
\include{end}
\fi